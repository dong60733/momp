\documentclass{ctexart}

% 插入宏包
\usepackage{graphicx} % Required for inserting images
\usepackage{geometry} % 设置页边距
\usepackage{lipsum} % 生成虚拟文本
\usepackage{fancyhdr} % 自定义页眉和页脚
\usepackage{booktabs} % 插入三线表
\usepackage{lastpage} % 解决总页数显示问题
\usepackage{amsmath,amsfonts,amsthm} % 常用数学公式指令、数学公式、提供证明环境
\usepackage{bm} % 数学字体加粗
\usepackage{mathrsfs} % 提供特殊的数学花体
\usepackage{amssymb, ,hyperref, framed, color, enumerate}

% Custom counter for problems
\newcounter{problemname}

% Environment for problems
\definecolor{shadecolor}{RGB}{241, 241, 255}

\newenvironment{problem}{\begin{shaded}\stepcounter{problemname}\par\noindent\textbf{习题}\arabic{problemname}.}{\end{shaded}\par}


% Environment for solutions
\newenvironment{solution}{\par\noindent\textbf{解答. }}{\par}

% Environment for notes
\newenvironment{note}{\par\noindent\textbf{习题\arabic{problemname}的注记. }}{\par}

% Reset problem counter at each subsection
\usepackage{titlesec}
\titleformat{\subsection}{\normalfont\large\bfseries}{\thesubsection}{1em}{\setcounter{problemname}{0}}

% 设置文章格式
\geometry{left=2.5cm,right=2.5cm,top=3cm,bottom=3cm}
\pagestyle{fancy} % 使用fancyhdr宏包定义页眉页脚
\fancyhf{} % 清空默认的页眉和页脚设置
\chead{数学物理方法作业}
\cfoot{第 \thepage 页(共 \pageref{LastPage}页)}

% 信息栏
\title{\Huge\textbf{数学物理方法作业}}
\author{Charles Luo}
\date{\today}

% 正文区
\begin{document}
\maketitle
\newpage
\tableofcontents
\newpage

\section{第一章习题}

\begin{problem}
计算下列表达式的值:
\begin{enumerate}[(1)]
    \item $\displaystyle(\frac{1+i}{2-i})^2\ $;
    \item $\displaystyle(1+i)^n+(1-i)^n\ $,\, 其中\ $n$\ 为整数.
\end{enumerate}
\end{problem}
\begin{solution}
    \begin{enumerate}[(1)]
        \item 原式 $=\displaystyle(\frac{(1+i)(2+i)}{(2-i)(2+i)})^2$
              $=\displaystyle(\frac{1+3i}{5})^2$
              $=\displaystyle\frac{-8+6i}{25}$\,.
        \item 由于$1+i=\sqrt{2}\text{e}^{\frac{\pi}{4}i}\ ,
                  1-i=\sqrt{2}\text{e}^{-\frac{\pi}{4}i}$\ .
              原式=$\displaystyle 2^{\frac{n}{2}}\text{e}^{\frac{n\pi}{4}i}
                  +2^{\frac{n}{2}}\text{e}^{-\frac{n\pi}{4}i}$
              =$\displaystyle 2^{\frac{n}{2}+1}\cos{\frac{n\pi}{4}}$\,.
    \end{enumerate}
\end{solution}

\begin{problem}
写出下列复数的实部、虚部、模和辐角:
\begin{enumerate}[(1)]
    \item $\displaystyle 1+i\sqrt{3} $\ ;
    \item $\displaystyle \text{e}^{i\sin{x}}$\ , \ $x$\ 为实数;
    \item $\displaystyle \text{e}^{iz} $\ ;
    \item $\displaystyle \text{e}^z $\ ;
    \item $\displaystyle \text{e}^{i\phi (x)}$\ , $\phi (x)$是实变数\ $x$\ 的实函数;
    \item $\displaystyle 1-\cos{\alpha}+i\sin{\alpha}$\ , $0\leq \alpha <2\pi$\ .
\end{enumerate}
\end{problem}
\begin{solution}
    \begin{table}[h]
        \centering
        \begin{tabular}{ccccc}
            \toprule % 三线表划分线
            \qquad 题号 \qquad\qquad & \qquad\qquad 实部 \qquad\qquad        & \qquad\qquad 虚部 \qquad\qquad        & \qquad 模 \qquad\qquad                   & 辐角                     \\
            \midrule
            (1)                    & 1                                   & $\sqrt{3}$                          & 2                                       & $\dfrac{\pi}{3}+2k\pi$ \\
            (2)                    & $\cos{\sin{x}}$                     & $\sin{\sin{x}}$                     & 1                                       & $\sin{x}+2k\pi$        \\
            (3)                    & $\displaystyle\text{e}^{-y}\cos{x}$ & $\displaystyle\text{e}^{-y}\sin{x}$ & $\displaystyle\text{e}^{-y}$            & $x+2k\pi$              \\
            (4)                    & $\displaystyle\text{e}^x\cos{y}$    & $\displaystyle\text{e}^x\sin{y}$    & $\displaystyle\text{e}^x$               & $y+2k\pi$              \\
            (5)                    & $\displaystyle\cos{\phi(x)}$        & $\displaystyle\sin{\phi(x)}$        & 1                                       & $\phi(x)+2k\pi$        \\
            (6)                    & $\displaystyle1-\cos{\alpha}$       & $\displaystyle\sin{\alpha}$         & $\displaystyle2\sin{\dfrac{\alpha}{2}}$ &
            $\displaystyle\dfrac{\pi-\alpha}{2}+2k\pi$                                                                                                                            \\
            \bottomrule
        \end{tabular}
    \end{table}
\end{solution}
\begin{note}
    (3)(4)中$x$是$z$的实部,$y$是$z$的虚部。
\end{note}

\begin{problem}
把下列关系用几何图形表示出来:
\begin{enumerate}[(1)]
    \item $\left| z \right| < 2, \left| z \right| = 2,
              \left| z \right| > 2 ;$
    \item $ \text{Re} \, z > \dfrac{1}{2} ;$
    \item $ 1 < \text{Im} \, z < 2 ;$
    \item $ 0 < \text{arg}(1-z) < \dfrac{\pi}{4} ;$
    \item $ \left| z \right| + \text{Re} \, z < 1 ;$
    \item $ 0 < \text{arg}(\dfrac{z+1}{z-1}) < \dfrac{\pi}{4} ;$
    \item $ \left| z - a \right| = \left| z - b \right| ,$\,$a,b$\,为常数;
    \item $ \left| z - a \right| + \left| z - b \right| = c $,
          其中\,$a,b,c$\,均为常数,$c>\left|a-b\right|$.
\end{enumerate}
\end{problem}
\begin{solution}
    \begin{enumerate}[(1)]
        \item 以原点为圆心画一个半径为$2$的圆,表示区域分别是圆内、圆上和圆外。
        \item 在实轴\ $\dfrac{1}{2}$处画一条平行于虚轴的直线,所求为直线右边区域。
        \item 在虚轴$1$和$2$处分别画一条平行于实轴的直线,所求为两直线之间区域。
        \item 由于\ $z=x+yi$\ ,故\ $1-z=(1-x)-yi$\ ,根据题意有\ $1-x > 0$\ ,
              $0<\dfrac{-y}{1-x}<1$\ ,解\ $x<1$\ ,$x-1<y<0$。
        \item 由于\ $z=x+yi$\ ,根据题意\ $x+\sqrt{x^2+y^2}<1$\ ,化简得到
              \ $y^2<1-2x$。
        \item 由于\ $z=x+yi$\ ,根据题意\ $\displaystyle\frac{x+1+yi}{x-1+yi}$\ 可以
              化简为\ $\dfrac{x^2+y^2-1}{x^2-2x+y^2+1}-\dfrac{2yi}{x^2-2x+y^2+1}$\ ,
              而辐角范围为\ $(0,\dfrac{\pi}{4})$\ ,有\ $x^2+y^2-1>0$\ ,
              $0<\dfrac{-2y}{x^2+y^2-1}<1$\ ,画出来的图像是\ $y<0$\ 部分挖去以\ $(0,-1)$ \
              为圆心,$\sqrt{2}$为半径的圆。
        \item 根据题意,点到\ $a$,$b$\ 的距离相等,点在$ab$连线的中垂线上。
        \item 根据题意,点到\ $a$,$b$\ 的距离和为定值,符合椭圆定义,故点在以\ $a$,$b$\ 为焦点的椭圆上。
    \end{enumerate}
\end{solution}

\newpage
\section{第二章习题}

\begin{problem}
判断下列函数在何处可导(并求出其导函数),在何处解析:
\begin{enumerate}[(1)]
    \item $\left|z\right|$\ ;
    \item $z^*$\ ;
    \item $z\text{Re}\ z$\ ;
    \item $(x^2+2y)+i(x^2+y^2)$\ ;
    \item $3x^2+2iy^2$\ ;
    \item $(x-y)^2+2i(x+y)$\ .
\end{enumerate}
\end{problem}
\begin{solution}
    \begin{enumerate}[(1)]
        \item 由于\ $z=x+iy$,$f(z)=\sqrt{x^2+y^2}$,
              \begin{equation*}
                  \dfrac{\partial u}{\partial x} = \dfrac{x}{\sqrt{x^2+y^2}}
              \end{equation*}
              \begin{equation*}
                  \dfrac{\partial u}{\partial y} = \dfrac{y}{\sqrt{x^2+y^2}}
              \end{equation*}
              \begin{equation*}
                  \dfrac{\partial v}{\partial x} = 0
              \end{equation*}
              \begin{equation*}
                  \dfrac{\partial v}{\partial y} = 0
              \end{equation*}
              若满足C-R方程,则\ $x=y=0$,
              而沿着\ $y=x$\ 趋近原点时,
              \begin{equation*}
                  \dfrac{\partial f}{\partial x} = \dfrac{\sqrt{2}}{2} \neq 0
              \end{equation*}
              故处处不可导,不解析。
        \item 若可导,则有\ $\dfrac{\partial f}{\partial z^*} = 0$,故处处不可导,不解析。
        \item 由于\ $z=x+iy$,$f(z)= x^2+ixy$,
              \begin{equation*}
                  \dfrac{\partial u}{\partial x} = 2x
              \end{equation*}
              \begin{equation*}
                  \dfrac{\partial u}{\partial y} = 0
              \end{equation*}
              \begin{equation*}
                  \dfrac{\partial v}{\partial x} = y
              \end{equation*}
              \begin{equation*}
                  \dfrac{\partial v}{\partial y} = x
              \end{equation*}
              若满足C-R方程,则\ $x=y=0$,
              现令\ $x=\rho\sin{\theta},y=\rho\cos{\theta}$,
              \begin{equation*}
                  \dfrac{\partial f}{\partial z} = \lim\limits_{\rho\to 0}
                  \dfrac{\rho^2\cos{\theta}^2+i\rho^2\sin{\theta}\cos{\theta}}{\rho\cos{\theta}+i\rho\sin{\theta}}=\rho\cos{\theta}=0
              \end{equation*}
              故仅在$(0,0)$处可导,不解析。
        \item 由题可以得到
              \begin{equation*}
                  \dfrac{\partial u}{\partial x} = 2x
              \end{equation*}
              \begin{equation*}
                  \dfrac{\partial u}{\partial y} = 2
              \end{equation*}
              \begin{equation*}
                  \dfrac{\partial v}{\partial x} = 2x
              \end{equation*}
              \begin{equation*}
                  \dfrac{\partial v}{\partial y} = 2y
              \end{equation*}
              若满足C-R方程,则\ $y=x$,$x=-1$,
              现令\ $x=\rho\sin{\theta},y=\rho\cos{\theta}$,
              \begin{equation*}
                  \dfrac{\partial f}{\partial z} = \lim\limits_{\rho\to 0}
              \end{equation*}

    \end{enumerate}
\end{solution}
\end{document}