\documentclass{ctexart}

% 插入宏包
\usepackage{graphicx} % Required for inserting images
\usepackage{geometry} % 设置页边距
\usepackage{lipsum} % 生成虚拟文本
\usepackage{fancyhdr} % 自定义页眉和页脚
\usepackage{booktabs} % 插入三线表
\usepackage{lastpage} % 解决总页数显示问题
\usepackage{amsmath,amsfonts,amsthm} % 常用数学公式指令、数学公式、提供证明环境
\usepackage{bm} % 数学字体加粗
\usepackage{mathrsfs} % 提供特殊的数学花体
\usepackage{amssymb, ,hyperref, framed, color, enumerate}

% Custom counter for problems
\newcounter{problemname}

% Environment for problems
\definecolor{shadecolor}{RGB}{241, 241, 255}

\newenvironment{problem}{\begin{shaded}\stepcounter{problemname}\par\noindent\textbf{习题}\arabic{problemname}.}{\end{shaded}\par}


% Environment for solutions
\newenvironment{solution}{\par\noindent\textbf{解答. }}{\par}

% Environment for notes
\newenvironment{note}{\par\noindent\textbf{习题\arabic{problemname}的注记. }}{\par}

% Reset problem counter at each subsection
\usepackage{titlesec}
\titleformat{\subsection}{\normalfont\large\bfseries}{\thesubsection}{1em}{\setcounter{problemname}{0}}

% 设置文章格式
\geometry{left=2.5cm,right=2.5cm,top=3cm,bottom=3cm}
\pagestyle{fancy} % 使用fancyhdr宏包定义页眉页脚
\fancyhf{} % 清空默认的页眉和页脚设置
\linespread{1.2}
\chead{数学物理方法作业}
\cfoot{第 \thepage 页(共 \pageref{LastPage}页)}

% 信息栏
\title{\Huge\textbf{数学物理方法作业}}
\author{Charles Luo}
\date{\today}

% 正文区
\begin{document}
\maketitle
\newpage
\tableofcontents
\newpage

\section{第一章习题}

\begin{problem}
计算下列表达式的值:
\begin{enumerate}[(1)]
    \item $\displaystyle(\frac{1+i}{2-i})^2\ $;
    \item $\displaystyle(1+i)^n+(1-i)^n\ $,\, 其中\ $n$\ 为整数.
\end{enumerate}
\end{problem}
\begin{solution}
    \begin{enumerate}[(1)]
        \item 原式 $=\displaystyle(\frac{(1+i)(2+i)}{(2-i)(2+i)})^2$
              $=\displaystyle(\frac{1+3i}{5})^2$
              $=\displaystyle\frac{-8+6i}{25}$\,.
        \item 由于$1+i=\sqrt{2}\text{e}^{\frac{\pi}{4}i}\ ,
                  1-i=\sqrt{2}\text{e}^{-\frac{\pi}{4}i}$\ .
              原式=$\displaystyle 2^{\frac{n}{2}}\text{e}^{\frac{n\pi}{4}i}
                  +2^{\frac{n}{2}}\text{e}^{-\frac{n\pi}{4}i}$
              =$\displaystyle 2^{\frac{n}{2}+1}\cos{\frac{n\pi}{4}}$\,.
    \end{enumerate}
\end{solution}

\begin{problem}
写出下列复数的实部、虚部、模和辐角:
\begin{enumerate}[(1)]
    \item $\displaystyle 1+i\sqrt{3} $\ ;
    \item $\displaystyle \text{e}^{i\sin{x}}$\ , \ $x$\ 为实数;
    \item $\displaystyle \text{e}^{iz} $\ ;
    \item $\displaystyle \text{e}^z $\ ;
    \item $\displaystyle \text{e}^{i\phi (x)}$\ , $\phi (x)$是实变数\ $x$\ 的实函数;
    \item $\displaystyle 1-\cos{\alpha}+i\sin{\alpha}$\ , $0\leq \alpha <2\pi$\ .
\end{enumerate}
\end{problem}
\begin{solution}
    \begin{table}[h]
        \centering
        \begin{tabular}{ccccc}
            \toprule % 三线表划分线
            \qquad 题号 \qquad\qquad & \qquad\qquad 实部 \qquad\qquad        & \qquad\qquad 虚部 \qquad\qquad        & \qquad 模 \qquad\qquad                   & 辐角                     \\
            \midrule
            (1)                    & 1                                   & $\sqrt{3}$                          & 2                                       & $\dfrac{\pi}{3}+2k\pi$ \\
            (2)                    & $\cos{\sin{x}}$                     & $\sin{\sin{x}}$                     & 1                                       & $\sin{x}+2k\pi$        \\
            (3)                    & $\displaystyle\text{e}^{-y}\cos{x}$ & $\displaystyle\text{e}^{-y}\sin{x}$ & $\displaystyle\text{e}^{-y}$            & $x+2k\pi$              \\
            (4)                    & $\displaystyle\text{e}^x\cos{y}$    & $\displaystyle\text{e}^x\sin{y}$    & $\displaystyle\text{e}^x$               & $y+2k\pi$              \\
            (5)                    & $\displaystyle\cos{\phi(x)}$        & $\displaystyle\sin{\phi(x)}$        & 1                                       & $\phi(x)+2k\pi$        \\
            (6)                    & $\displaystyle1-\cos{\alpha}$       & $\displaystyle\sin{\alpha}$         & $\displaystyle2\sin{\dfrac{\alpha}{2}}$ &
            $\displaystyle\dfrac{\pi-\alpha}{2}+2k\pi$                                                                                                                            \\
            \bottomrule
        \end{tabular}
    \end{table}
\end{solution}
\begin{note}
    (3)(4)中$x$是$z$的实部,$y$是$z$的虚部。
\end{note}

\begin{problem}
把下列关系用几何图形表示出来:
\begin{enumerate}[(1)]
    \item $\left| z \right| < 2, \left| z \right| = 2,
              \left| z \right| > 2 ;$
    \item $ \text{Re} \, z > \dfrac{1}{2} ;$
    \item $ 1 < \text{Im} \, z < 2 ;$
    \item $ 0 < \text{arg}(1-z) < \dfrac{\pi}{4} ;$
    \item $ \left| z \right| + \text{Re} \, z < 1 ;$
    \item $ 0 < \text{arg}(\dfrac{z+1}{z-1}) < \dfrac{\pi}{4} ;$
    \item $ \left| z - a \right| = \left| z - b \right| ,$\,$a,b$\,为常数;
    \item $ \left| z - a \right| + \left| z - b \right| = c $,
          其中\,$a,b,c$\,均为常数,$c>\left|a-b\right|$.
\end{enumerate}
\end{problem}
\begin{solution}
    \begin{enumerate}[(1)]
        \item 以原点为圆心画一个半径为$2$的圆,表示区域分别是圆内、圆上和圆外。
        \item 在实轴\ $\dfrac{1}{2}$处画一条平行于虚轴的直线,所求为直线右边区域。
        \item 在虚轴$1$和$2$处分别画一条平行于实轴的直线,所求为两直线之间区域。
        \item 由于\ $z=x+yi$\ ,故\ $1-z=(1-x)-yi$\ ,根据题意有\ $1-x > 0$\ ,
              $0<\dfrac{-y}{1-x}<1$\ ,解\ $x<1$\ ,$x-1<y<0$。
        \item 由于\ $z=x+yi$\ ,根据题意\ $x+\sqrt{x^2+y^2}<1$\ ,化简得到
              \ $y^2<1-2x$。
        \item 由于\ $z=x+yi$\ ,根据题意\ $\displaystyle\frac{x+1+yi}{x-1+yi}$\ 可以
              化简为\ $\dfrac{x^2+y^2-1}{x^2-2x+y^2+1}-\dfrac{2yi}{x^2-2x+y^2+1}$\ ,
              而辐角范围为\ $(0,\dfrac{\pi}{4})$\ ,有\ $x^2+y^2-1>0$\ ,
              $0<\dfrac{-2y}{x^2+y^2-1}<1$\ ,画出来的图像是\ $y<0$\ 部分挖去以\ $(0,-1)$ \
              为圆心,$\sqrt{2}$为半径的圆。
        \item 根据题意,点到\ $a$,$b$\ 的距离相等,点在$ab$连线的中垂线上。
        \item 根据题意,点到\ $a$,$b$\ 的距离和为定值,符合椭圆定义,故点在以\ $a$,$b$\ 为焦点的椭圆上。
    \end{enumerate}
\end{solution}

\newpage
\section{第二章习题}

\begin{problem}
判断下列函数在何处可导(并求出其导函数),在何处解析:
\begin{enumerate}[(1)]
    \item $\left|z\right|$\ ;
    \item $z^*$\ ;
    \item $z\text{Re}\ z$\ ;
    \item $(x^2+2y)+i(x^2+y^2)$\ ;
    \item $3x^2+2iy^2$\ ;
    \item $(x-y)^2+2i(x+y)$\ .
\end{enumerate}
\end{problem}
\begin{solution}
    \begin{enumerate}[(1)]
        \item 由于\ $z=x+iy$,$f(z)=\sqrt{x^2+y^2}$,
              \begin{equation*}
                  \dfrac{\partial u}{\partial x} = \dfrac{x}{\sqrt{x^2+y^2}}
              \end{equation*}
              \begin{equation*}
                  \dfrac{\partial u}{\partial y} = \dfrac{y}{\sqrt{x^2+y^2}}
              \end{equation*}
              \begin{equation*}
                  \dfrac{\partial v}{\partial x} = 0
              \end{equation*}
              \begin{equation*}
                  \dfrac{\partial v}{\partial y} = 0
              \end{equation*}
              若满足C-R方程,则\ $x=y=0$,
              而沿着\ $y=x$\ 趋近原点时,
              \begin{equation*}
                  \dfrac{\partial f}{\partial x} = \dfrac{\sqrt{2}}{2} \neq 0
              \end{equation*}
              故处处不可导,不解析。
        \item 若可导,则有\ $\dfrac{\partial f}{\partial z^*} = 0$,故处处不可导,不解析。
        \item 由于\ $z=x+iy$,$f(z)= x^2+ixy$,
              \begin{equation*}
                  \dfrac{\partial u}{\partial x} = 2x
              \end{equation*}
              \begin{equation*}
                  \dfrac{\partial u}{\partial y} = 0
              \end{equation*}
              \begin{equation*}
                  \dfrac{\partial v}{\partial x} = y
              \end{equation*}
              \begin{equation*}
                  \dfrac{\partial v}{\partial y} = x
              \end{equation*}
              若满足C-R方程,则\ $x=y=0$,
              现令\ $x=\rho\sin{\theta},y=\rho\cos{\theta}$,
              \begin{equation*}
                  \dfrac{\partial f}{\partial z} = \lim\limits_{\rho\to 0}
                  \dfrac{\rho^2\cos{\theta}^2+i\rho^2\sin{\theta}\cos{\theta}}{\rho\cos{\theta}+i\rho\sin{\theta}}=\rho\cos{\theta}=0
              \end{equation*}
              故仅在$(0,0)$处可导,不解析。
        \item 由题可以得到
              \begin{equation*}
                  \dfrac{\partial u}{\partial x} = 2x
              \end{equation*}
              \begin{equation*}
                  \dfrac{\partial u}{\partial y} = 2
              \end{equation*}
              \begin{equation*}
                  \dfrac{\partial v}{\partial x} = 2x
              \end{equation*}
              \begin{equation*}
                  \dfrac{\partial v}{\partial y} = 2y
              \end{equation*}
              若满足C-R方程,则\ $y=x$,$x=-1$,
              现令\ $x=\rho\sin{\theta},y=\rho\cos{\theta}$,
              \begin{equation*}
                  \dfrac{\partial f}{\partial z} = \lim\limits_{\rho\to 0}\dfrac{\rho^2\cos{\theta}^2-2\rho\cos{\theta}+2\rho\sin{\theta}+i\rho^2-2i\rho\cos{\theta}-2i\rho\sin{\theta}}{\rho\cos{\theta}+i\sin{\theta}}
              \end{equation*}
              \begin{equation*}
                  = \lim\limits_{\rho\to 0}\dfrac{-2\cos{\theta}+2\sin{\theta}-2i\cos{\theta}-2i\sin{\theta}}{\cos{\theta}+i\sin{\theta}}
                  = -2 - 2i
              \end{equation*}
              故仅在$(-1,1)$处可导,导数为\ $-2-2i$\ ,不解析。
        \item 由题可以得到
              \begin{equation*}
                  \dfrac{\partial u}{\partial x} = 6x
              \end{equation*}
              \begin{equation*}
                  \dfrac{\partial u}{\partial y} = 0
              \end{equation*}
              \begin{equation*}
                  \dfrac{\partial v}{\partial x} = 0
              \end{equation*}
              \begin{equation*}
                  \dfrac{\partial v}{\partial y} = 6y^2
              \end{equation*}
              若满足C-R方程,则\ $x=y^2$,此时\ $f(z)=3y^4+2iy^2$,
              \begin{equation*}
                  \dfrac{\partial f}{\partial z} = \dfrac{\partial v}{\partial y}-i\dfrac{\partial u}{\partial y} = 6y^2
              \end{equation*}
              故在\ $x=y^2$\ 上可导,导数为\ $6y^2$,不解析。
        \item 由题可以得到
              \begin{equation*}
                  \dfrac{\partial u}{\partial x} = 2x - 2y
              \end{equation*}
              \begin{equation*}
                  \dfrac{\partial u}{\partial y} = 2y - 2x
              \end{equation*}
              \begin{equation*}
                  \dfrac{\partial v}{\partial x} = 2
              \end{equation*}
              \begin{equation*}
                  \dfrac{\partial v}{\partial y} = 2
              \end{equation*}
              若满足C-R方程,则\ $2x-2y=2$\ 即\ $x=y+1$,此时\ $f(z)=1+i(4y+2)$,
              \begin{equation*}
                  \dfrac{\partial f}{\partial z} = \dfrac{\partial v}{\partial y}-i\dfrac{\partial u}{\partial y} = 2 + 2i
              \end{equation*}
              故在\ $x=y+1$\ 上可导,导数为\ $2+2i$,不解析。
    \end{enumerate}
\end{solution}

\begin{problem}
设\ $z=x+iy$,已知解析函数\ $f(z)=u(x,y)+iv(x,y)$\ 的实部或虚部如下,试求\
$f'(z)$\ :
\begin{enumerate}[(1)]
    \item $u=x+y$\ ;
    \item $u=\sin{x}\cosh{y}$\ .
\end{enumerate}
\end{problem}
\begin{solution}
    \begin{enumerate}[(1)]
        \item 由函数解析可知C-R方程成立,而$\dfrac{\partial u}{\partial x} = \dfrac{\partial u}{\partial y} = 1$,故$\dfrac{\partial v}{\partial x} = -1$,$\dfrac{\partial v}{\partial y} = 1$. \\[15pt]
              于是可以求出 $\displaystyle v(x,y) = \int_{(0,0)}^{(x,0)}-\text{d}x + \int_{(x,0)}^{(x,t)}\text{d}y = -x + y + C$. \\[15pt]
              即 $\displaystyle f(z) = x + y + i(y - x) + iC = z - iz + iC$, $\displaystyle f'(z) = 1 - i$。
        \item 由函数解析可知C-R方程成立,而$\dfrac{\partial u}{\partial x} = \cos{x}\cosh{y}$\ ,\ $\dfrac{\partial u}{\partial y} = \sin{x}\sinh{y}$,\\[15pt]
              故$\dfrac{\partial v}{\partial x} = -\sin{x}\sinh{y}$\ ,$\dfrac{\partial v}{\partial y} = \cos{x}\cosh{y}$\ 。 \\[15pt]
              于是可以求出$\displaystyle v(x,y) = \int_{(0,0)}^{(x,0)}-\sin{x}\sinh{0}\text{d}x + \int_{(x,0)}^{(x,t)}\cos{x}\cosh{y}\text{d}y = \cos{x}\sinh{y} + C$. \\[15pt]
              即 $\displaystyle f(z) = \sin{x}\cosh{y} + i\cos{x}\sinh{y} + iC$, $\displaystyle f'(z) = \dfrac{\partial u}{\partial x} + \dfrac{\partial v}{\partial x} = \cos{x}\cosh{y} - \sin{x}\sinh{y}\textcolor{red}{=\cos{z}}$。
    \end{enumerate}
\end{solution}
\begin{note}
    \begin{itemize}
        \item $\cos{z} = \dfrac{\text{e}^{iz}+\text{e}^{-iz}}{2}$
        \item $\sin{z} = \dfrac{\text{e}^{iz}-\text{e}^{-iz}}{2i}$
        \item $\sinh{z} = \dfrac{\text{e}^{z}-\text{e}^{-z}}{2}$
        \item $\cosh{z} = \dfrac{\text{e}^{z}+\text{e}^{-z}}{2}$
        \item $\sinh{z} = -i\sin{iz}$
        \item $\cosh{z} = \cos{iz}$
    \end{itemize}
\end{note}

\begin{problem}
若\ $f(z)=u(x,y)+iv(x,y)$\ 解析,且\ $u-v=(x-y)(x^2+4xy+y^2)$,试\ $f(z)$\ .
\end{problem}
\begin{solution}
    由题,
    \begin{equation*}
        \dfrac{\partial u}{\partial x} - \dfrac{\partial v}{\partial x} = x^2 + 4xy + y^2 + (x-y)(2x+4y)\ ,
    \end{equation*}
    \begin{equation*}
        \dfrac{\partial u}{\partial y} - \dfrac{\partial v}{\partial y} = -(x^2 + 4xy + y^2) + (x-y)(4x + 2y)\ .
    \end{equation*}
    解析函数满足C-R方程,即 $\dfrac{\partial u}{\partial x} = \dfrac{\partial v}{\partial y}$\ , \ $\dfrac{\partial u}{\partial y} = \dfrac{\partial v}{\partial x}$\ . \\[15pt]
    解出$\dfrac{\partial u}{\partial x} = \dfrac{\partial v}{\partial y} = 6xy$\ , \ $\dfrac{\partial u}{\partial y} = \dfrac{\partial v}{\partial x} = 3(x^2 - y^2)$\ . \\[15pt]
    $\displaystyle u(x,y) = \int_{(0,0)}^{(x,0)}0\text{d}x + \int_{(x,0)}^{(x,t)}3(x^2-y^2)\text{d}y = 3x^2y - y^2 + C_1$. \\[15pt]
    $\displaystyle v(x,y) = \int_{(0,0)}^{(x,0)}-3x^2\text{d}x + \int_{(x,0)}^{(x,t)}6xy\text{d}y = -x^3 + 3xy^2 + C_2$. \\[15pt]
    而 $u-v$ 中不含常数,故 $C_1 = C_2 = C$\ , \\[15pt]
    $f(z) = u + iv = 3x^2y - y^3 + i(3xy^2 - x^3) + (1+i)C\textcolor{red}{= iz^3 + (1+i)C}$
\end{solution}

\begin{problem}
判断下列哪些是函数,哪些是多值函数:
\begin{enumerate}[(1)]
    \item $\sqrt{z^2-1}$\ ;
    \item $z+\sqrt{z-1}$\ ;
    \item $\sin{\sqrt{z}}$\ ;
    \item $\cos{\sqrt{z}}$\ ;
    \item $\dfrac{\sin{\sqrt{z}}}{\sqrt{z}}$\ ;
    \item $\dfrac{\cos{\sqrt{z}}}{\sqrt{z}}$\ ;
    \item $\ln{\sin{z}}$\ ;
    \item $\sin{(i\ln{z})}$\ ;
\end{enumerate}
\end{problem}
\begin{solution}
    \begin{enumerate}[(1)]
        \item 多值函数。
        \item 多值函数。
        \item 已知$\sqrt{z} = \pm \omega$\ ,\ 且$\sin{\omega} \neq \sin{-\omega}$\ ,\ 故为多值函数。
        \item 虽然$\sqrt{z} = \pm \omega$\ ,\ 但是$\cos{\omega} = \cos{-\omega}$\ ,\ 故为单值函数。
        \item 虽然$\sqrt{z} = \pm \omega$\ ,\ 但是$\dfrac{\sin{\omega}}{\omega} = \dfrac{\sin{(-\omega)}}{-\omega}$\ ,\ 故为单值函数。
        \item 已知$\sqrt{z} = \pm \omega$\ ,\ 且$\dfrac{\cos{\omega}}{\omega} \neq \dfrac{\cos{(-\omega)}}{-\omega}$\ ,\ 故为多值函数。
        \item 多值函数。
        \item 已知 $\ln{z}$ 是多值函数,对应的函数值满足关系的是值相同,幅角相差$2\pi$的整数倍,而正弦函数又以$2\pi$为周期,故为单值函数。
    \end{enumerate}
\end{solution}

\begin{problem}
找出下列多值函数的分支点,并讨论$z$绕一个分支点移动一周回到原点处后多值函数值的变化。如果同时绕两个、三个乃至更多个分支点一周,多值函数的值又如何变化?
\begin{enumerate}[(1)]
    \item $\sqrt{(z-a)(z-b)}$\ ,\ $a\neq b$\ ;
    \item $\sqrt[3]{(z-a)(z-b)}$\ ,\ $a\neq b$\ ;
    \item $\sqrt{1-z^3}$\ ;
    \item $\sqrt[3]{1-z^3}$\ ;
    \item $\ln{(z^2+1)}$\ ;
    \item $\ln{\cos{z}}$\ ;
\end{enumerate}
\end{problem}
\begin{solution}
    \begin{enumerate}[(1)]
        \item 枝点可能为 $a$,\ $b$,\ $\infty$\ ,逐一验证:
              \begin{itemize}
                  \item 令$z = a + \epsilon\text{e}^{i\varphi},\quad \epsilon\to 0,\ \varphi\in(0,2\pi)$,此时 $f(z) = \text{e}^{\frac{1}{2}i\varphi}\sqrt{(a - b)\epsilon}$. \\
                        显然 $\varphi = 0$和$\varphi = 2\pi$时函数值不等,故 $a$ 为枝点。
                  \item 同理,$b$也为枝点。
                  \item 现考虑 $\infty$,做变换 $t = \dfrac{1}{z}$,令 $t = \epsilon\text{e}^{i\varphi},\quad \epsilon\to 0,\ \varphi\in(0,2\pi)$,此时 $f(\infty) = \text{e}^{-i\varphi}\sqrt{\dfrac{1}{\epsilon^2}}$. \\
                        显然 $\varphi = 0$和$\varphi = 2\pi$时函数值相等,故 $\infty$ 不是枝点。
              \end{itemize}
              故枝点为 $a$,\ $b$。
        \item 枝点可能为 $a$,\ $b$,\ $\infty$\ ,逐一验证:
              \begin{itemize}
                  \item 令$z = a + \epsilon\text{e}^{i\varphi},\quad \epsilon\to 0,\ \varphi\in(0,2\pi)$,此时 $f(z) = \text{e}^{\frac{1}{3}i\varphi}\sqrt[3]{(a - b)\epsilon}$. \\
                        显然 $\varphi = 0$和$\varphi = 2\pi$时函数值不等,故 $a$ 为枝点。
                  \item 同理,$b$也为枝点。
                  \item 现考虑 $\infty$,做变换 $t = \dfrac{1}{z}$,令 $t = \epsilon\text{e}^{i\varphi},\quad \epsilon\to 0,\ \varphi\in(0,2\pi)$,此时 $f(\infty) = \text{e}^{-\frac{2}{3}i\varphi}\sqrt[3]{\dfrac{1}{\epsilon^2}}$. \\
                        显然 $\varphi = 0$和$\varphi = 2\pi$时函数值不等,故 $\infty$ 为枝点。
              \end{itemize}
              故枝点为 $a$,\ $b$,\ $\infty$。
        \item 因式分解得 $\sqrt{(1-z)(z-\text{e}^{i\frac{2\pi}{3}})(z-\text{e}^{-i\frac{2\pi}{3}})}$,故猜测枝点为 $1$,\ $\text{e}^{i\frac{2\pi}{3}}$,\ $\text{e}^{-i\frac{2\pi}{3}}$,\ $\infty$,逐一验证:
              \begin{itemize}
                  \item 令$z = 1 + \epsilon\text{e}^{i\varphi},\quad \epsilon\to 0,\ \varphi\in(0,2\pi)$,此时 $f(z) = \text{e}^{\frac{1}{2}i\varphi}\sqrt{(1-\text{e}^{i\frac{2\pi}{3}})(1-\text{e}^{-i\frac{2\pi}{3}})\epsilon}$. \\
                        显然 $\varphi = 0$和$\varphi = 2\pi$时函数值不等,故 $1$ 为枝点。
                  \item 同理,$\text{e}^{i\frac{2\pi}{3}}$也为枝点。
                  \item 同理,$\text{e}^{-i\frac{2\pi}{3}}$也为枝点。
                  \item 现考虑 $\infty$,做变换 $t = \dfrac{1}{z}$,令 $t = \epsilon\text{e}^{i\varphi},\quad \epsilon\to 0,\ \varphi\in(0,2\pi)$,此时 $f(\infty) = \text{e}^{-\frac{3}{2}i\varphi}\sqrt{\dfrac{1}{\epsilon^3}}$. \\
                        显然 $\varphi = 0$和$\varphi = 2\pi$时函数值不等,故 $\infty$ 为枝点。
              \end{itemize}
              故枝点为  $1$,\ $\text{e}^{i\frac{2\pi}{3}}$,\ $\text{e}^{-i\frac{2\pi}{3}}$,\ $\infty$。
        \item 因式分解得 $\sqrt[3]{(1-z)(z-\text{e}^{i\frac{2\pi}{3}})(z-\text{e}^{-i\frac{2\pi}{3}})}$,故猜测枝点为 $1$,\ $\text{e}^{i\frac{2\pi}{3}}$,\ $\text{e}^{-i\frac{2\pi}{3}}$,\ $\infty$,逐一验证:
              \begin{itemize}
                  \item 令$z = 1 + \epsilon\text{e}^{i\varphi},\quad \epsilon\to 0,\ \varphi\in(0,2\pi)$,此时 $f(z) = \text{e}^{\frac{1}{3}i\varphi}\sqrt[3]{(1-\text{e}^{i\frac{2\pi}{3}})(1-\text{e}^{-i\frac{2\pi}{3}})\epsilon}$. \\
                        显然 $\varphi = 0$和$\varphi = 2\pi$时函数值不等,故 $1$ 为枝点。
                  \item 同理,$\text{e}^{i\frac{2\pi}{3}}$也为枝点。
                  \item 同理,$\text{e}^{-i\frac{2\pi}{3}}$也为枝点。
                  \item 现考虑 $\infty$,做变换 $t = \dfrac{1}{z}$,令 $t = \epsilon\text{e}^{i\varphi},\quad \epsilon\to 0,\ \varphi\in(0,2\pi)$,此时 $f(\infty) = \text{e}^{-i\varphi}\sqrt[3]{\dfrac{1}{\epsilon^3}}$. \\
                        显然 $\varphi = 0$和$\varphi = 2\pi$时函数值相等,故 $\infty$ 不是枝点。
              \end{itemize}
              故枝点为  $1$,\ $\text{e}^{i\frac{2\pi}{3}}$,\ $\text{e}^{-i\frac{2\pi}{3}}$。
        \item 枝点可能为 $i$,\ $-i$,\ $\infty$,逐一验证:
              \begin{itemize}
                  \item 令$z = i + \epsilon\text{e}^{i\varphi},\quad \epsilon\to 0,\ \varphi\in(0,2\pi)$,此时 $f(z) = \ln{2i\epsilon\text{e}^{i\varphi}} = i\varphi + \ln{2i\epsilon}$. \\
                        显然 $\varphi = 0$和$\varphi = 2\pi$时函数值不等,故 $i$ 为枝点。
                  \item 同理,$-i$也为枝点。
                  \item 现考虑 $\infty$,做变换 $t = \dfrac{1}{z}$,令 $t = \epsilon\text{e}^{i\varphi},\quad \epsilon\to 0,\ \varphi\in(0,2\pi)$,此时 $f(\infty) = -i\varphi + \ln{\dfrac{1}{\epsilon}}$. \\
                        显然 $\varphi = 0$和$\varphi = 2\pi$时函数值不等,故 $\infty$ 为枝点。
                        故枝点为 $i$,\ $-i$,\ $\infty$ 。
              \end{itemize}
        \item 由 $\cos{z} = 0$可以解出 $z = \pm \dfrac{2n+1}{2}\pi, \quad n\in\mathbb{N}$,猜测这些根都是枝点。不妨以 $\dfrac{\pi}{2}$为例,令$z = \dfrac{\pi}{2} + \epsilon\text{e}^{i\varphi},\quad \epsilon\to 0,\ \varphi\in(0,2\pi)$,此时 $\displaystyle f(z) = \ln{\dfrac{\text{e}^{i(\frac{\pi}{2}+ \epsilon\text{e}^{i\varphi})}+\text{e}^{-i(\frac{\pi}{2}+ \epsilon\text{e}^{i\varphi})}}{2}} = \ln{\epsilon} + i\varphi$. \\
              显然 $\varphi = 0$和$\varphi = 2\pi$时函数值不等,故 $\infty$ 为枝点。\\[15pt]
              故枝点为 $z = \pm \dfrac{2n+1}{2}\pi, \quad n\in\mathbb{N}$ 。
    \end{enumerate}
\end{solution}

\end{document}