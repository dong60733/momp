\documentclass{ctexart}

% 插入宏包
\usepackage{graphicx} % Required for inserting images
\usepackage{geometry} % 设置页边距
\usepackage{lipsum} % 生成虚拟文本
\usepackage{fancyhdr} % 自定义页眉和页脚
\usepackage{booktabs} % 插入三线表
\usepackage{lastpage} % 解决总页数显示问题
\usepackage{amsmath,amsfonts,amsthm} % 常用数学公式指令、数学公式、提供证明环境
\usepackage{bm} % 数学字体加粗
\usepackage{mathrsfs} % 提供特殊的数学花体
\usepackage{amssymb, ,hyperref, framed, color, enumerate}

% Custom counter for problems
\newcounter{problemname}

% Environment for problems
\definecolor{shadecolor}{RGB}{241, 241, 255}

\newenvironment{problem}{\begin{shaded}\stepcounter{problemname}\par\noindent\textbf{习题}\arabic{problemname}.}{\end{shaded}\par}


% Environment for solutions
\newenvironment{solution}{\par\noindent\textbf{解答. }}{\par}

% Environment for notes
\newenvironment{note}{\par\noindent\textbf{习题\arabic{problemname}的注记. }}{\par}

% Reset problem counter at each subsection
\usepackage{titlesec}
\titleformat{\subsection}{\normalfont\large\bfseries}{\thesubsection}{1em}{\setcounter{problemname}{0}}

% 设置文章格式
\geometry{left=2.5cm,right=2.5cm,top=3cm,bottom=3cm}
\pagestyle{fancy} % 使用fancyhdr宏包定义页眉页脚
\fancyhf{} % 清空默认的页眉和页脚设置
\chead{数学物理方法作业}
\cfoot{第 \thepage 页(共 \pageref{LastPage}页)}

% 信息栏
\title{\Huge\textbf{数学物理方法作业}}
\author{Charles Luo}
\date{\today}

% 正文区
\begin{document}
\maketitle
\newpage
\tableofcontents
\newpage

\section{第一章习题}

\begin{problem}
计算下列表达式的值:
\begin{enumerate}[(1)]
    \item $\displaystyle(\frac{1+i}{2-i})^2\ $;
    \item $\displaystyle(1+i)^n+(1-i)^n\ $,\, 其中\ $n$\ 为整数.
\end{enumerate}
\end{problem}

\begin{solution}
    \begin{enumerate}[(1)]
        \item 原式 $=\displaystyle(\frac{(1+i)(2+i)}{(2-i)(2+i)})^2$
              $=\displaystyle(\frac{1+3i}{5})^2$
              $=\displaystyle\frac{-8+6i}{25}$\,.
        \item 由于$1+i=\sqrt{2}\text{e}^{\frac{\pi}{4}i}\ ,
                  1-i=\sqrt{2}\text{e}^{-\frac{\pi}{4}i}$\ .
              原式=$\displaystyle 2^{\frac{n}{2}}\text{e}^{\frac{n\pi}{4}i}
                  +2^{\frac{n}{2}}\text{e}^{-\frac{n\pi}{4}i}$
              =$\displaystyle 2^{\frac{n}{2}+1}\cos{\frac{n\pi}{4}}$\,.
    \end{enumerate}
\end{solution}

\begin{problem}
写出下列复数的实部、虚部、模和辐角:
\begin{enumerate}
    \item $\displaystyle 1+i\sqrt{3} $\ ;
    \item $\displaystyle \text{e}^{i\sin{x}}$\ , \ $x$\ 为实数;
    \item $\displaystyle \text{e}^{iz} $\ ;
    \item $\displaystyle \text{e}^z $\ ;
    \item $\displaystyle \text{e}^{i\phi (x)}$\ , $\phi (x)$是实变数\ $x$\ 的实函数;
    \item $\displaystyle 1-\cos{\alpha}+i\sin{\alpha}$\ , $0\leq \alpha <2\pi$\ .
\end{enumerate}
\end{problem}

\begin{solution}
    \begin{table}[h]
        \centering
        \begin{tabular}{ccccc}
            \toprule % 三线表划分线
            \qquad 题号 \qquad\qquad & \qquad\qquad 实部 \qquad\qquad & \qquad\qquad 虚部 \qquad\qquad & \qquad 模 \qquad\qquad & 辐角                     \\
            \midrule
            (1)            & 1                            & $\sqrt{3}$                   & 2                           & $\dfrac{\pi}{3}+2k\pi$ \\
            (2)            & $\cos{\sin{x}}$              & $\sin{\sin{x}}$              & 1                           & $\sin{x}+2k\pi$        \\
            \bottomrule
        \end{tabular}
    \end{table}
\end{solution}

\end{document}