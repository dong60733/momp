\documentclass[11pt]{ctexart}

% 插入宏包
\usepackage{graphicx} % Required for inserting images
\usepackage{geometry} % 设置页边距
\usepackage{lipsum} % 生成虚拟文本
\usepackage{fancyhdr} 
\setlength{\headheight}{14pt}% 自定义页眉和页脚
\usepackage{booktabs} % 插入三线表
\usepackage{lastpage} % 解决总页数显示问题
\usepackage{amsmath,amsfonts,amsthm} % 常用数学公式指令、数学公式、提供证明环境
\usepackage{bm} % 数学字体加粗
\usepackage{mathrsfs} % 提供特殊的数学花体
\usepackage{amssymb, ,hyperref, framed, color, enumerate}

% Custom counter for problems
\newcounter{problemname}

% Environment for problems
\definecolor{shadecolor}{RGB}{241, 241, 255}

\newenvironment{problem}{\begin{shaded}\stepcounter{problemname}\par\noindent\textbf{习题}\arabic{problemname}.}{\end{shaded}\par}


% Environment for solutions
\newenvironment{solution}{\par\noindent\textbf{解答. }}{\par}

% Environment for notes
\newenvironment{note}{\par\noindent\textbf{习题\arabic{problemname}的注记. }}{\par}

% Reset problem counter at each subsection
\usepackage{titlesec}
\titleformat{\subsection}{\normalfont\large\bfseries}{\thesubsection}{1em}{\setcounter{problemname}{0}}

% 设置文章格式
\geometry{left=2.5cm,right=2.5cm,top=3cm,bottom=3cm}
\pagestyle{fancy} % 使用fancyhdr宏包定义页眉页脚
\fancyhf{} % 清空默认的页眉和页脚设置
\linespread{1.5}
\chead{数学物理方法作业}
\cfoot{第 \thepage 页(共 \pageref{LastPage}页)}

% 信息栏
\title{\Huge\textbf{数学物理方法作业}}
\author{Charles Luo}
\date{\today}

% 正文区
\begin{document}
\maketitle
\newpage
\tableofcontents
\newpage

\section{第一章习题}

\begin{problem}
计算下列表达式的值:
\begin{enumerate}[(1)]
    \item $\displaystyle(\frac{1+i}{2-i})^2\ $;
    \item $\displaystyle(1+i)^n+(1-i)^n\ $,\, 其中\ $n$\ 为整数.
\end{enumerate}
\end{problem}
\begin{solution}
    \begin{enumerate}[(1)]
        \item 原式 $=\displaystyle(\frac{(1+i)(2+i)}{(2-i)(2+i)})^2$
              $=\displaystyle(\frac{1+3i}{5})^2$
              $=\displaystyle\frac{-8+6i}{25}$\,.
        \item 由于$1+i=\sqrt{2}\text{e}^{\frac{\pi}{4}i}\ ,
                  1-i=\sqrt{2}\text{e}^{-\frac{\pi}{4}i}$\ .
              原式=$\displaystyle 2^{\frac{n}{2}}\text{e}^{\frac{n\pi}{4}i}
                  +2^{\frac{n}{2}}\text{e}^{-\frac{n\pi}{4}i}$
              =$\displaystyle 2^{\frac{n}{2}+1}\cos{\frac{n\pi}{4}}$\,.
    \end{enumerate}
\end{solution}

\begin{problem}
写出下列复数的实部、虚部、模和辐角:
\begin{enumerate}[(1)]
    \item $\displaystyle 1+i\sqrt{3} $\ ;
    \item $\displaystyle \text{e}^{i\sin{x}}$\ , \ $x$\ 为实数;
    \item $\displaystyle \text{e}^{iz} $\ ;
    \item $\displaystyle \text{e}^z $\ ;
    \item $\displaystyle \text{e}^{i\phi (x)}$\ , $\phi (x)$是实变数\ $x$\ 的实函数;
    \item $\displaystyle 1-\cos{\alpha}+i\sin{\alpha}$\ , $0\leq \alpha <2\pi$\ .
\end{enumerate}
\end{problem}
\begin{solution}
    \begin{table}[h]
        \centering
        \begin{tabular}{ccccc}
            \toprule % 三线表划分线
            \qquad 题号 \qquad\qquad & \qquad\qquad 实部 \qquad\qquad        & \qquad\qquad 虚部 \qquad\qquad        & \qquad 模 \qquad\qquad                   & 辐角                     \\
            \midrule
            (1)                    & 1                                   & $\sqrt{3}$                          & 2                                       & $\dfrac{\pi}{3}+2k\pi$ \\
            (2)                    & $\cos{\sin{x}}$                     & $\sin{\sin{x}}$                     & 1                                       & $\sin{x}+2k\pi$        \\
            (3)                    & $\displaystyle\text{e}^{-y}\cos{x}$ & $\displaystyle\text{e}^{-y}\sin{x}$ & $\displaystyle\text{e}^{-y}$            & $x+2k\pi$              \\
            (4)                    & $\displaystyle\text{e}^x\cos{y}$    & $\displaystyle\text{e}^x\sin{y}$    & $\displaystyle\text{e}^x$               & $y+2k\pi$              \\
            (5)                    & $\displaystyle\cos{\phi(x)}$        & $\displaystyle\sin{\phi(x)}$        & 1                                       & $\phi(x)+2k\pi$        \\
            (6)                    & $\displaystyle1-\cos{\alpha}$       & $\displaystyle\sin{\alpha}$         & $\displaystyle2\sin{\dfrac{\alpha}{2}}$ &
            $\displaystyle\dfrac{\pi-\alpha}{2}+2k\pi$                                                                                                                            \\
            \bottomrule
        \end{tabular}
    \end{table}
\end{solution}
\begin{note}
    (3)(4)中$x$是$z$的实部,$y$是$z$的虚部。
\end{note}

\begin{problem}
把下列关系用几何图形表示出来:
\begin{enumerate}[(1)]
    \item $\left| z \right| < 2, \left| z \right| = 2,
              \left| z \right| > 2 ;$
    \item $ \text{Re} \, z > \dfrac{1}{2} ;$
    \item $ 1 < \text{Im} \, z < 2 ;$
    \item $ 0 < \text{arg}(1-z) < \dfrac{\pi}{4} ;$
    \item $ \left| z \right| + \text{Re} \, z < 1 ;$
    \item $ 0 < \text{arg}(\dfrac{z+1}{z-1}) < \dfrac{\pi}{4} ;$
    \item $ \left| z - a \right| = \left| z - b \right| ,$\,$a,b$\,为常数;
    \item $ \left| z - a \right| + \left| z - b \right| = c $,
          其中\,$a,b,c$\,均为常数,$c>\left|a-b\right|$.
\end{enumerate}
\end{problem}
\begin{solution}
    \begin{enumerate}[(1)]
        \item 以原点为圆心画一个半径为$2$的圆,表示区域分别是圆内、圆上和圆外。
        \item 在实轴\ $\dfrac{1}{2}$处画一条平行于虚轴的直线,所求为直线右边区域。
        \item 在虚轴$1$和$2$处分别画一条平行于实轴的直线,所求为两直线之间区域。
        \item 由于\ $z=x+yi$\ ,故\ $1-z=(1-x)-yi$\ ,根据题意有\ $1-x > 0$\ ,
              $0<\dfrac{-y}{1-x}<1$\ ,解\ $x<1$\ ,$x-1<y<0$。
        \item 由于\ $z=x+yi$\ ,根据题意\ $x+\sqrt{x^2+y^2}<1$\ ,化简得到
              \ $y^2<1-2x$。
        \item 由于\ $z=x+yi$\ ,根据题意\ $\displaystyle\frac{x+1+yi}{x-1+yi}$\ 可以
              化简为\ $\dfrac{x^2+y^2-1}{x^2-2x+y^2+1}-\dfrac{2yi}{x^2-2x+y^2+1}$\ ,
              而辐角范围为\ $(0,\dfrac{\pi}{4})$\ ,有\ $x^2+y^2-1>0$\ ,
              $0<\dfrac{-2y}{x^2+y^2-1}<1$\ ,画出来的图像是\ $y<0$\ 部分挖去以\ $(0,-1)$ \
              为圆心,$\sqrt{2}$为半径的圆。
        \item 根据题意,点到\ $a$,$b$\ 的距离相等,点在$ab$连线的中垂线上。
        \item 根据题意,点到\ $a$,$b$\ 的距离和为定值,符合椭圆定义,故点在以\ $a$,$b$\ 为焦点的椭圆上。
    \end{enumerate}
\end{solution}

\newpage
\section{第二章习题}

\begin{problem}
判断下列函数在何处可导(并求出其导函数),在何处解析:
\begin{enumerate}[(1)]
    \item $\left|z\right|$\ ;
    \item $z^*$\ ;
    \item $z\text{Re}\ z$\ ;
    \item $(x^2+2y)+i(x^2+y^2)$\ ;
    \item $3x^2+2iy^2$\ ;
    \item $(x-y)^2+2i(x+y)$\ .
\end{enumerate}
\end{problem}
\begin{solution}
    \begin{enumerate}[(1)]
        \item 由于\ $z=x+iy$,$f(z)=\sqrt{x^2+y^2}$,
              \begin{equation*}
                  \dfrac{\partial u}{\partial x} = \dfrac{x}{\sqrt{x^2+y^2}}
              \end{equation*}
              \begin{equation*}
                  \dfrac{\partial u}{\partial y} = \dfrac{y}{\sqrt{x^2+y^2}}
              \end{equation*}
              \begin{equation*}
                  \dfrac{\partial v}{\partial x} = 0
              \end{equation*}
              \begin{equation*}
                  \dfrac{\partial v}{\partial y} = 0
              \end{equation*}
              若满足C-R方程,则\ $x=y=0$,
              而沿着\ $y=x$\ 趋近原点时,
              \begin{equation*}
                  \dfrac{\partial f}{\partial x} = \dfrac{\sqrt{2}}{2} \neq 0
              \end{equation*}
              故处处不可导,不解析。
        \item 若可导,则有\ $\dfrac{\partial f}{\partial z^*} = 0$,故处处不可导,不解析。
        \item 由于\ $z=x+iy$,$f(z)= x^2+ixy$,
              \begin{equation*}
                  \dfrac{\partial u}{\partial x} = 2x
              \end{equation*}
              \begin{equation*}
                  \dfrac{\partial u}{\partial y} = 0
              \end{equation*}
              \begin{equation*}
                  \dfrac{\partial v}{\partial x} = y
              \end{equation*}
              \begin{equation*}
                  \dfrac{\partial v}{\partial y} = x
              \end{equation*}
              若满足C-R方程,则\ $x=y=0$,
              现令\ $x=\rho\sin{\theta},y=\rho\cos{\theta}$,
              \begin{equation*}
                  \dfrac{\partial f}{\partial z} = \lim\limits_{\rho\to 0}
                  \dfrac{\rho^2\cos{\theta}^2+i\rho^2\sin{\theta}\cos{\theta}}{\rho\cos{\theta}+i\rho\sin{\theta}}=\rho\cos{\theta}=0
              \end{equation*}
              故仅在$(0,0)$处可导,不解析。
        \item 由题可以得到
              \begin{equation*}
                  \dfrac{\partial u}{\partial x} = 2x
              \end{equation*}
              \begin{equation*}
                  \dfrac{\partial u}{\partial y} = 2
              \end{equation*}
              \begin{equation*}
                  \dfrac{\partial v}{\partial x} = 2x
              \end{equation*}
              \begin{equation*}
                  \dfrac{\partial v}{\partial y} = 2y
              \end{equation*}
              若满足C-R方程,则\ $y=x$,$x=-1$,
              现令\ $x=\rho\sin{\theta},y=\rho\cos{\theta}$,
              \begin{equation*}
                  \dfrac{\partial f}{\partial z} = \lim\limits_{\rho\to 0}\dfrac{\rho^2\cos{\theta}^2-2\rho\cos{\theta}+2\rho\sin{\theta}+i\rho^2-2i\rho\cos{\theta}-2i\rho\sin{\theta}}{\rho\cos{\theta}+i\sin{\theta}} 
              \end{equation*}
              \begin{equation*}
                  = \lim\limits_{\rho\to 0}\dfrac{-2\cos{\theta}+2\sin{\theta}-2i\cos{\theta}-2i\sin{\theta}}{\cos{\theta}+i\sin{\theta}}
                  = -2 - 2i
              \end{equation*}
              故仅在$(-1,1)$处可导,导数为\ $-2-2i$\ ,不解析。
        \item 由题可以得到
              \begin{equation*}
                  \dfrac{\partial u}{\partial x} = 6x
              \end{equation*}
              \begin{equation*}
                  \dfrac{\partial u}{\partial y} = 0
              \end{equation*}
              \begin{equation*}
                  \dfrac{\partial v}{\partial x} = 0
              \end{equation*}
              \begin{equation*}
                  \dfrac{\partial v}{\partial y} = 6y^2
              \end{equation*}
              若满足C-R方程,则\ $x=y^2$,此时\ $f(z)=3y^4+2iy^2$,
              \begin{equation*}
                  \dfrac{\partial f}{\partial z} = \dfrac{\partial v}{\partial y}-i\dfrac{\partial u}{\partial y} = 6y^2
              \end{equation*}
              故在\ $x=y^2$\ 上可导,导数为\ $6y^2$,不解析。
        \item 由题可以得到
              \begin{equation*}
                  \dfrac{\partial u}{\partial x} = 2x - 2y
              \end{equation*}
              \begin{equation*}
                  \dfrac{\partial u}{\partial y} = 2y - 2x
              \end{equation*}
              \begin{equation*}
                  \dfrac{\partial v}{\partial x} = 2
              \end{equation*}
              \begin{equation*}
                  \dfrac{\partial v}{\partial y} = 2
              \end{equation*}
              若满足C-R方程,则\ $2x-2y=2$\ 即\ $x=y+1$,此时\ $f(z)=1+i(4y+2)$,
              \begin{equation*}
                  \dfrac{\partial f}{\partial z} = \dfrac{\partial v}{\partial y}-i\dfrac{\partial u}{\partial y} = 2 + 2i
              \end{equation*}
              故在\ $x=y+1$\ 上可导,导数为\ $2+2i$,不解析。
    \end{enumerate}
\end{solution}

\begin{problem}
    设\ $z=x+iy$,已知解析函数\ $f(z)=u(x,y)+iv(x,y)$\ 的实部或虚部如下,试求\ 
    $f'(z)$\ :
    \begin{enumerate}[(1)]
        \item $u=x+y$\ ;
        \item $u=\sin{x}\cosh{y}$\ .
    \end{enumerate}
\end{problem}
\begin{solution}
    \begin{enumerate}[(1)]
        \item 由函数解析可知C-R方程成立,而$\dfrac{\partial u}{\partial x} = \dfrac{\partial u}{\partial y} = 1$,故$\dfrac{\partial v}{\partial x} = -1$,$\dfrac{\partial v}{\partial y} = 1$. \\[15pt]
        于是可以求出 $\displaystyle v(x,y) = \int_{(0,0)}^{(x,0)}-\text{d}x + \int_{(x,0)}^{(x,t)}\text{d}y = -x + y + C$. \\[15pt]
        即 $\displaystyle f(z) = x + y + i(y - x) + iC = z - iz + iC$, $\displaystyle f'(z) = 1 - i$。
        \item 由函数解析可知C-R方程成立,而$\dfrac{\partial u}{\partial x} = \cos{x}\cosh{y}$\ ,\ $\dfrac{\partial u}{\partial y} = \sin{x}\sinh{y}$,\\[15pt]
        故$\dfrac{\partial v}{\partial x} = -\sin{x}\sinh{y}$\ ,$\dfrac{\partial v}{\partial y} = \cos{x}\cosh{y}$\ 。 \\[15pt]
        于是可以求出$\displaystyle v(x,y) = \int_{(0,0)}^{(x,0)}-\sin{x}\sinh{0}\text{d}x + \int_{(x,0)}^{(x,t)}\cos{x}\cosh{y}\text{d}y = \cos{x}\sinh{y} + C$. \\[15pt]
        即 $\displaystyle f(z) = \sin{x}\cosh{y} + i\cos{x}\sinh{y} + iC$, $\displaystyle f'(z) = \dfrac{\partial u}{\partial x} + \dfrac{\partial v}{\partial x} = \cos{x}\cosh{y} - \sin{x}\sinh{y}\textcolor{red}{=\cos{z}}$。
    \end{enumerate}
\end{solution}
\begin{note}
    \begin{itemize}
        \item $\cos{z} = \dfrac{\text{e}^{iz}+\text{e}^{-iz}}{2}$
        \item $\sin{z} = \dfrac{\text{e}^{iz}-\text{e}^{-iz}}{2i}$
        \item $\sinh{z} = \dfrac{\text{e}^{z}-\text{e}^{-z}}{2}$
        \item $\cosh{z} = \dfrac{\text{e}^{z}+\text{e}^{-z}}{2}$
        \item $\sinh{z} = -i\sin{iz}$
        \item $\cosh{z} = \cos{iz}$
    \end{itemize}
\end{note}

\begin{problem}
    若\ $f(z)=u(x,y)+iv(x,y)$\ 解析,且\ $u-v=(x-y)(x^2+4xy+y^2)$,试\ $f(z)$\ .
\end{problem}
\begin{solution}
    由题,
    \begin{equation*}
        \dfrac{\partial u}{\partial x} - \dfrac{\partial v}{\partial x} = x^2 + 4xy + y^2 + (x-y)(2x+4y)\ ,
    \end{equation*}
    \begin{equation*}
        \dfrac{\partial u}{\partial y} - \dfrac{\partial v}{\partial y} = -(x^2 + 4xy + y^2) + (x-y)(4x + 2y)\ .
    \end{equation*}
    解析函数满足C-R方程,即 $\dfrac{\partial u}{\partial x} = \dfrac{\partial v}{\partial y}$\ , \ $\dfrac{\partial u}{\partial y} = \dfrac{\partial v}{\partial x}$\ . \\[15pt]
    解出$\dfrac{\partial u}{\partial x} = \dfrac{\partial v}{\partial y} = 6xy$\ , \ $\dfrac{\partial u}{\partial y} = \dfrac{\partial v}{\partial x} = 3(x^2 - y^2)$\ . \\[15pt]
    $\displaystyle u(x,y) = \int_{(0,0)}^{(x,0)}0\text{d}x + \int_{(x,0)}^{(x,t)}3(x^2-y^2)\text{d}y = 3x^2y - y^2 + C_1$. \\[15pt]
    $\displaystyle v(x,y) = \int_{(0,0)}^{(x,0)}-3x^2\text{d}x + \int_{(x,0)}^{(x,t)}6xy\text{d}y = -x^3 + 3xy^2 + C_2$. \\[15pt]
    而 $u-v$ 中不含常数,故 $C_1 = C_2 = C$\ , \\[15pt]
    $f(z) = u + iv = 3x^2y - y^3 + i(3xy^2 - x^3) + (1+i)C\textcolor{red}{= iz^3 + (1+i)C}$
\end{solution}

\begin{problem}
    判断下列哪些是函数,哪些是多值函数:
    \begin{enumerate}[(1)]
        \item $\sqrt{z^2-1}$\ ;
        \item $z+\sqrt{z-1}$\ ;
        \item $\sin{\sqrt{z}}$\ ;
        \item $\cos{\sqrt{z}}$\ ;
        \item $\dfrac{\sin{\sqrt{z}}}{\sqrt{z}}$\ ;
        \item $\dfrac{\cos{\sqrt{z}}}{\sqrt{z}}$\ ;
        \item $\ln{\sin{z}}$\ ;
        \item $\sin{(i\ln{z})}$\ ;
    \end{enumerate}
\end{problem}
\begin{solution}
    \begin{enumerate}[(1)]
        \item 多值函数。
        \item 多值函数。
        \item 已知$\sqrt{z} = \pm \omega$\ ,\ 且$\sin{\omega} \neq \sin{-\omega}$\ ,\ 故为多值函数。
        \item 虽然$\sqrt{z} = \pm \omega$\ ,\ 但是$\cos{\omega} = \cos{-\omega}$\ ,\ 故为单值函数。
        \item 虽然$\sqrt{z} = \pm \omega$\ ,\ 但是$\dfrac{\sin{\omega}}{\omega} = \dfrac{\sin{(-\omega)}}{-\omega}$\ ,\ 故为单值函数。
        \item 已知$\sqrt{z} = \pm \omega$\ ,\ 且$\dfrac{\cos{\omega}}{\omega} \neq \dfrac{\cos{(-\omega)}}{-\omega}$\ ,\ 故为多值函数。
        \item 多值函数。
        \item 已知 $\ln{z}$ 是多值函数,对应的函数值满足关系的是值相同,幅角相差$2\pi$的整数倍,而正弦函数又以$2\pi$为周期,故为单值函数。
    \end{enumerate}
\end{solution}

\begin{problem}
    找出下列多值函数的分支点,并讨论$z$绕一个分支点移动一周回到原点处后多值函数值的变化。如果同时绕两个、三个乃至更多个分支点一周,多值函数的值又如何变化?
    \begin{enumerate}[(1)]
        \item $\sqrt{(z-a)(z-b)}$\ ,\ $a\neq b$\ ;
        \item $\sqrt[3]{(z-a)(z-b)}$\ ,\ $a\neq b$\ ;
        \item $\sqrt{1-z^3}$\ ;
        \item $\sqrt[3]{1-z^3}$\ ;
        \item $\ln{(z^2+1)}$\ ;
        \item $\ln{\cos{z}}$\ ;
    \end{enumerate}
\end{problem}
\begin{solution}
    \begin{enumerate}[(1)]
        \item 枝点可能为 $a$,\ $b$,\ $\infty$\ ,逐一验证:
        \begin{itemize}
            \item 令$z = a + \epsilon\text{e}^{i\varphi},\quad \epsilon\to 0,\ \varphi\in(0,2\pi)$,此时 $f(z) = \text{e}^{\frac{1}{2}i\varphi}\sqrt{(a - b)\epsilon}$. \\
            显然 $\varphi = 0$和$\varphi = 2\pi$时函数值不等,故 $a$ 为枝点。
            \item 同理,$b$也为枝点。
            \item 现考虑 $\infty$,做变换 $t = \dfrac{1}{z}$,令 $t = \epsilon\text{e}^{i\varphi},\quad \epsilon\to 0,\ \varphi\in(0,2\pi)$,此时 $f(\infty) = \text{e}^{-i\varphi}\sqrt{\dfrac{1}{\epsilon^2}}$. \\
            显然 $\varphi = 0$和$\varphi = 2\pi$时函数值相等,故 $\infty$ 不是枝点。
        \end{itemize}
        故枝点为 $a$,\ $b$。
        \item 枝点可能为 $a$,\ $b$,\ $\infty$\ ,逐一验证:
        \begin{itemize}
            \item 令$z = a + \epsilon\text{e}^{i\varphi},\quad \epsilon\to 0,\ \varphi\in(0,2\pi)$,此时 $f(z) = \text{e}^{\frac{1}{3}i\varphi}\sqrt[3]{(a - b)\epsilon}$. \\
            显然 $\varphi = 0$和$\varphi = 2\pi$时函数值不等,故 $a$ 为枝点。
            \item 同理,$b$也为枝点。
            \item 现考虑 $\infty$,做变换 $t = \dfrac{1}{z}$,令 $t = \epsilon\text{e}^{i\varphi},\quad \epsilon\to 0,\ \varphi\in(0,2\pi)$,此时 $f(\infty) = \text{e}^{-\frac{2}{3}i\varphi}\sqrt[3]{\dfrac{1}{\epsilon^2}}$. \\
            显然 $\varphi = 0$和$\varphi = 2\pi$时函数值不等,故 $\infty$ 为枝点。
        \end{itemize}
        故枝点为 $a$,\ $b$,\ $\infty$。
        \item 因式分解得 $\sqrt{(1-z)(z-\text{e}^{i\frac{2\pi}{3}})(z-\text{e}^{-i\frac{2\pi}{3}})}$,故猜测枝点为 $1$,\ $\text{e}^{i\frac{2\pi}{3}}$,\ $\text{e}^{-i\frac{2\pi}{3}}$,\ $\infty$,逐一验证:
        \begin{itemize}
            \item 令$z = 1 + \epsilon\text{e}^{i\varphi},\quad \epsilon\to 0,\ \varphi\in(0,2\pi)$,此时 $f(z) = \text{e}^{\frac{1}{2}i\varphi}\sqrt{(1-\text{e}^{i\frac{2\pi}{3}})(1-\text{e}^{-i\frac{2\pi}{3}})\epsilon}$. \\
            显然 $\varphi = 0$和$\varphi = 2\pi$时函数值不等,故 $1$ 为枝点。
            \item 同理,$\text{e}^{i\frac{2\pi}{3}}$也为枝点。
            \item 同理,$\text{e}^{-i\frac{2\pi}{3}}$也为枝点。
            \item 现考虑 $\infty$,做变换 $t = \dfrac{1}{z}$,令 $t = \epsilon\text{e}^{i\varphi},\quad \epsilon\to 0,\ \varphi\in(0,2\pi)$,此时 $f(\infty) = \text{e}^{-\frac{3}{2}i\varphi}\sqrt{\dfrac{1}{\epsilon^3}}$. \\
            显然 $\varphi = 0$和$\varphi = 2\pi$时函数值不等,故 $\infty$ 为枝点。
        \end{itemize}
        故枝点为  $1$,\ $\text{e}^{i\frac{2\pi}{3}}$,\ $\text{e}^{-i\frac{2\pi}{3}}$,\ $\infty$。
        \item 因式分解得 $\sqrt[3]{(1-z)(z-\text{e}^{i\frac{2\pi}{3}})(z-\text{e}^{-i\frac{2\pi}{3}})}$,故猜测枝点为 $1$,\ $\text{e}^{i\frac{2\pi}{3}}$,\ $\text{e}^{-i\frac{2\pi}{3}}$,\ $\infty$,逐一验证:
        \begin{itemize}
            \item 令$z = 1 + \epsilon\text{e}^{i\varphi},\quad \epsilon\to 0,\ \varphi\in(0,2\pi)$,此时 $f(z) = \text{e}^{\frac{1}{3}i\varphi}\sqrt[3]{(1-\text{e}^{i\frac{2\pi}{3}})(1-\text{e}^{-i\frac{2\pi}{3}})\epsilon}$. \\
            显然 $\varphi = 0$和$\varphi = 2\pi$时函数值不等,故 $1$ 为枝点。
            \item 同理,$\text{e}^{i\frac{2\pi}{3}}$也为枝点。
            \item 同理,$\text{e}^{-i\frac{2\pi}{3}}$也为枝点。
            \item 现考虑 $\infty$,做变换 $t = \dfrac{1}{z}$,令 $t = \epsilon\text{e}^{i\varphi},\quad \epsilon\to 0,\ \varphi\in(0,2\pi)$,此时 $f(\infty) = \text{e}^{-i\varphi}\sqrt[3]{\dfrac{1}{\epsilon^3}}$. \\
            显然 $\varphi = 0$和$\varphi = 2\pi$时函数值相等,故 $\infty$ 不是枝点。
        \end{itemize}
        故枝点为  $1$,\ $\text{e}^{i\frac{2\pi}{3}}$,\ $\text{e}^{-i\frac{2\pi}{3}}$。
        \item 枝点可能为 $i$,\ $-i$,\ $\infty$,逐一验证:
    \begin{itemize}
        \item 令$z = i + \epsilon\text{e}^{i\varphi},\quad \epsilon\to 0,\ \varphi\in(0,2\pi)$,此时 $f(z) = \ln{2i\epsilon\text{e}^{i\varphi}} = i\varphi + \ln{2i\epsilon}$. \\
        显然 $\varphi = 0$和$\varphi = 2\pi$时函数值不等,故 $i$ 为枝点。
        \item 同理,$-i$也为枝点。
        \item 现考虑 $\infty$,做变换 $t = \dfrac{1}{z}$,令 $t = \epsilon\text{e}^{i\varphi},\quad \epsilon\to 0,\ \varphi\in(0,2\pi)$,此时 $f(\infty) = -i\varphi + \ln{\dfrac{1}{\epsilon}}$. \\
        显然 $\varphi = 0$和$\varphi = 2\pi$时函数值不等,故 $\infty$ 为枝点。
        故枝点为 $i$,\ $-i$,\ $\infty$ 。
    \end{itemize}
    \item 由 $\cos{z} = 0$可以解出 $z = \pm \dfrac{2n+1}{2}\pi, \quad n\in\mathbb{N}$,猜测这些根都是枝点。不妨以 $\dfrac{\pi}{2}$为例,令$z = \dfrac{\pi}{2} + \epsilon\text{e}^{i\varphi},\quad \epsilon\to 0,\ \varphi\in(0,2\pi)$,此时 $\displaystyle f(z) = \ln{\dfrac{\text{e}^{i(\frac{\pi}{2}+ \epsilon\text{e}^{i\varphi})}+\text{e}^{-i(\frac{\pi}{2}+ \epsilon\text{e}^{i\varphi})}}{2}} = \ln{\epsilon} + i\varphi$. \\
    显然 $\varphi = 0$和$\varphi = 2\pi$时函数值不等,故 $\infty$ 为枝点。\\
    故枝点为 $z = \pm \dfrac{2n+1}{2}\pi, \quad n\in\mathbb{N}$ 。
    \end{enumerate}
\end{solution}

\newpage
\section{第三章习题}

\begin{problem}
    试按给定的路径计算下列积分:
    \begin{enumerate}[(1)]
        \item $\displaystyle\int_0^{2+i}\text{Re}\ z\text{d}z$,积分路径为:
        \begin{enumerate}[(i)]
            \item 线段 $[0,2]$ 和 $[2,2+2i]$ 组成的折线.
            \item 线段 $z=(2+i)t,\quad 0<t\leq 1$.
        \end{enumerate}
        \item $\displaystyle\int_C \dfrac{\text{d}z}{\sqrt{z}}$, 规定 $\sqrt{z}|_{z=1}=1$,积分路径为由 $z=1$ 出发的:
        \begin{enumerate}[(i)]
            \item 单位圆的上半周.
            \item 单位圆的下半周.
        \end{enumerate}
    \end{enumerate}
\end{problem}
\begin{solution}
    \begin{enumerate}[(1)]
        \item \begin{enumerate}[(i)]
                  \item 由于 $z=x+iy$,$\text{d}z=\text{d}x+i\text{d}y$,故有\\
                        $\displaystyle\int_{0}^{2+i}\text{Re}\ z\text{d}z = \int_{0}^{2}x\text{d}x 
                        + \int_{0}^{1}2i\text{d}y = 2 + 2i$.
                  \item 此时 $x=2t$ , $y=t$ , $\text{d}z=(2+i)\text{d}t$,故有\\
                        $\displaystyle\int_{0}^{2+i}\text{Re}\ z\text{d}z = \int_{0}^{2+i}2t(2+i)\text{d}t 
                        = \int_{0}^{1}(4t+2it)\text{d}t = 2+i$.   
               \end{enumerate}
        \item 已知$z=\text{e}^{i\theta}$ , $\text{d}z=i\text{e}^{i\theta}\text{d}\theta$ , $\sqrt{z}=\text{e}^{\frac{i\theta}{2}}$
              \begin{enumerate}[(i)]
                   \item  $\displaystyle\int_C \dfrac{\text{d}z}{\sqrt{z}} = \int_{0}^{\pi}\text{e}^{-\frac{i\theta}{2}}i\text{e}^{i\theta}\text{d}\theta
                          = 2\int_{0}^{\pi}\text{e}^{\frac{i\theta}{2}\text{d}(\frac{i\theta}{2})}
                          = 2\text{e}^{\frac{i\theta}{2}}|_0^{\pi}
                          = 2i-2$.
                    \item $\displaystyle\int_C \dfrac{\text{d}z}{\sqrt{z}} = \int_{0}^{\pi}\text{e}^{-\frac{i\theta}{2}}i\text{e}^{i\theta}\text{d}\theta
                          = 2\int_{0}^{-\pi}\text{e}^{\frac{i\theta}{2}\text{d}(\frac{i\theta}{2})}
                          = 2\text{e}^{\frac{i\theta}{2}}|_0^{-\pi}
                          = -2i-2$.              
              \end{enumerate}
    \end{enumerate}
\end{solution}

\begin{problem}
    计算下列积分:
    \begin{enumerate}[(1)]
        \item $\displaystyle\oint_{\left| z \right| = 1}\dfrac{\text{d}z}{z}$;
        \item $\displaystyle\oint_{\left| z \right| = 1}\dfrac{\left|\text{d}z\right|}{z}$;
        \item $\displaystyle\oint_{\left| z \right| = 1}\dfrac{\text{d}z}{\left|z\right|}$;
        \item $\displaystyle\oint_{\left| z \right| = 1}\left|\dfrac{\text{d}z}{z}\right|$;
    \end{enumerate}
\end{problem}
\begin{solution}
    在单位圆上,有$z=\text{e}^{i\theta}$ ,$\text{d}z=i\text{e}^{i\theta}\text{d}\theta$.
    \begin{enumerate}[(1)]
        \item $\displaystyle\oint_{\left| z \right| = 1}\dfrac{\text{d}z}{z} = 
              \int_{0}^{2\pi}\text{e}^{-i\theta}i\text{e}^{i\theta}\text{d}\theta =
              2\pi i$;
        \item 此时$\left|\text{d}z\right|=\text{d}\theta$,故$\displaystyle\oint_{\left| z \right| = 1}\dfrac{\left|\text{d}z\right|}{z} =
              \int_{0}^{2\pi}\text{e}^{-i\theta}\text{d}\theta = 
              -\frac{1}{i}\int_{0}^{2\pi}\text{e}^{-i\theta}\text{d}(-i\theta) = 
              -\frac{1}{i}\text{e}^{i\theta}|_0^{2\pi} = 
              0$;
        \item 此时$\left|z\right|=1$,故$\displaystyle\oint_{\left| z \right| = 1}\dfrac{\text{d}z}{\left|z\right|} = 
              \int_{0}^{2\pi}i\text{e}^{i\theta}\text{d}\theta = 
              \text{e}^{i\theta}|_0^{2\pi} = 
              0$;
        \item $\displaystyle\oint_{\left| z \right| = 1}\left|\dfrac{\text{d}z}{z}\right| = 
              \int_{0}^{2\pi}\left|\text{e}^{-i\theta}i\text{e}^{i\theta}\text{d}\theta\right| = 
              2\pi$.
    \end{enumerate}
\end{solution}

\begin{problem}
    计算下列积分:
    \begin{enumerate}[(1)]
        \item $\displaystyle\oint_C \frac{1}{z^2-1}\sin{\frac{\pi z}{4}}\text{d}z$, $C$分别为:
        \begin{enumerate}[(i)]
            \item $\left|z\right|=\dfrac{1}{2}$.
            \item $\left|z\right|=3$.
        \end{enumerate}
        \item $\displaystyle\oint_C \frac{1}{z^2+1} \text{e}^{iz}\text{d}z$, $C$分别为:
        \begin{enumerate}[(i)]
            \item $\left|z-i\right|=1$.
            \item $\left|z+i\right|+\left|z-i\right|=2\sqrt{2}$.
        \end{enumerate}
    \end{enumerate}
\end{problem}
\begin{solution}
    \begin{enumerate}[(1)]
        \item 对被积函数分析,$\displaystyle f(z)=\frac{1}{(z+1)(z-1)}\sin{\frac{\pi z}{4}}$,故奇点为 $1$ 和 $-1$.
        \begin{enumerate}[(i)]
            \item 显然此时的围道不包含奇点,由Cauchy定理,积分结果为0。
            \item 此时积分积分围道包含奇点 $1$ 和 $-1$,由Cauchy积分公式,有\\
            $\displaystyle\oint_C\frac{1}{(z+1)(z-1)}\sin{\frac{\pi z}{4}}\text{d}z = 
            2\pi i(\frac{1}{z+1}\sin{\frac{\pi z}{4}})|_{z=1} + 2\pi i(\frac{1}{z-1}\sin{\frac{\pi z}{4}})|_{z=-1} = 
            \sqrt{2}\pi i$.
        \end{enumerate}
        \item 对被积函数分析,$\displaystyle f(z)=\frac{1}{(z+i)(z-i)}\text{e}^{iz}$,故奇点为 $i$ 和 $-i$.
        \begin{enumerate}[(i)]
            \item 此时包含奇点 $i$,由Cauchy积分公式,有\\
                  $\displaystyle\oint_C\frac{1}{(z+i)(z-i)}\text{e}^{iz}\text{d}z = 
                  2\pi i(\frac{1}{z+i}\text{e}^{iz})|_{z=i} = 
                  \frac{\pi}{\text{e}}$.
            \item 此时包含奇点 $i$ 和 $-i$,由Cauchy积分公式,有\\
                  $\displaystyle\oint_C\frac{1}{(z+i)(z-i)}\text{e}^{iz}\text{d}z =
                  2\pi i(\frac{1}{z+i}\text{e}^{iz})|_{z=i} + 2\pi i(\frac{1}{z-i}\text{e}^{iz})|_{z=-i} = 
                  \frac{\pi}{\text{e}} - \pi\text{e}\textcolor{red}{= -2\pi\sinh{1}}$.
        \end{enumerate} 
    \end{enumerate}
\end{solution}

\begin{problem}
    计算下列积分:
    \begin{enumerate}
        \item $\displaystyle\oint_{\left|z\right|=2} \frac{\cos{z}}{z}\text{d}z$;
        \item $\displaystyle\oint_{\left|z\right|=2} \frac{z^2-1}{z^2+1}\text{d}z$;
        \item $\displaystyle\oint_{\left|z\right|=2} \frac{\sin{\text{e}^z}}{z}\text{d}z$;
        \item $\displaystyle\oint_{\left|z\right|=2} \frac{\text{e}^z}{\cosh{z}}\text{d}z$;
        \item $\displaystyle\oint_{\left|z\right|=2} \frac{\sin{z}}{z^2}\text{d}z$;
        \item $\displaystyle\oint_{\left|z\right|=2} \frac{\left|z\right|\text{e}^z}{z^2}\text{d}z$;
        \item $\displaystyle\oint_{\left|z\right|=2} \frac{\sin{z}}{z^4}\text{d}z$;
        \item $\displaystyle\oint_{\left|z\right|=2}\frac{\text{d}z}{z^2(z^2+16)}$.
    \end{enumerate}
\end{problem}
\begin{solution}
    \begin{enumerate}[(1)]
        \item 奇点为原点,在围道内,由Cauchy积分公式,有\\
              $\displaystyle\oint_{\left|z\right|=2} \frac{\cos{z}}{z}\text{d}z = 
              2\pi i(\cos{z})|_{z=0} = 
              2\pi i$;
        \item 对被积函数分析,$\displaystyle f(z)=\frac{z^2-1}{(z+i)(z-i)}$,故奇点为 $i$ 和 $-i$,均在围道内,由Cauchy积分公式,有\\
              $\displaystyle\oint_{\left|z\right|=2} \frac{z^2-1}{z^2+1}\text{d}z = 
              2\pi i(\frac{z^2-1}{z+i})|_{z=i} + 2\pi i(\frac{z^2-1}{z-i})|_{z=-i} = 
              -2\pi + 2\pi = 
              0$;
        \item 奇点为原点,在围道内,由Cauchy积分公式,有\\
              $\displaystyle\oint_{\left|z\right|=2} \frac{\sin{\text{e}^z}}{z}\text{d}z = 
              2\pi i(\sin{\text{e}^z})|_{z=0} = 
              2\pi \sin{1} $;
        \item 对被积函数分析,$\displaystyle f(z)=\frac{\text{e}^{z}}{\cos{iz}}$,奇点为$\displaystyle\frac{\pi}{2}i+2k\pi i\ ,\ k\in\mathbb{Z}$,
              其中$\pm\dfrac{\pi i}{2}$在围道内,但此时不满足Cauchy积分公式所需表达形式,故应根据Cauchy定理,将原积分围道转化为两个围绕奇点的围道再求和,
              在 $\dfrac{\pi i}{2}$ 点附近选取一半径为 $\rho$ 的圆为围道 $C_1$,
              在 $-\dfrac{\pi i}{2}$ 点附近选取一半径为 $\rho$ 的圆为围道 $C_2$,
              先考虑$\displaystyle\oint_{C_1} \frac{\text{e}^z}{\cosh{z}}\text{d}z$,
              不妨取 $z=\dfrac{\pi i}{2}+\rho\text{e}^{i\theta}$,此时 $\text{d}z=i\rho\text{e}^{i\theta}\text{d}\theta$,则有
              \begin{equation*}
                \oint_{C_1} \frac{\text{e}^z}{\cosh{z}}\text{d}z = 
                \int_{0}^{2\pi}\frac{\text{e}^{\frac{\pi i}{2}+\rho\text{e}^{i\theta}}}{\cosh{(\frac{\pi i}{2}+\rho\text{e}^{i\theta})}}i\rho\text{e}^{i\theta}\text{d}\theta
              \end{equation*}
              当 $\rho\to0$ 时,且 $\cosh{z}=\cos{iz}$,可以化简得到
              \begin{equation*}
                \oint_{C_1} \frac{\text{e}^z}{\cosh{z}}\text{d}z = 
                \int_{0}^{2\pi}\frac{\text{e}^{\frac{\pi i}{2}}}{\cos{(-\frac{\pi}{2}+i\rho\text{e}^{i\theta})}}i\rho\text{e}^{i\theta}\text{d}\theta = 
                \int_{0}^{2\pi}\frac{\text{e}^{\frac{\pi i}{2}}}{i\rho\text{e}^{i\theta}}i\rho\text{e}^{i\theta}\text{d}\theta =
                2\pi i
              \end{equation*}
              再考虑$\displaystyle\oint_{C_2} \frac{\text{e}^z}{\cosh{z}}\text{d}z$,
              不妨取 $z=-\dfrac{\pi i}{2}+\rho\text{e}^{i\theta}$,此时 $\text{d}z=i\rho\text{e}^{i\theta}\text{d}\theta$,则有
              \begin{equation*}
                \oint_{C_2} \frac{\text{e}^z}{\cosh{z}}\text{d}z = 
                \int_{0}^{2\pi}\frac{\text{e}^{-\frac{\pi i}{2}+\rho\text{e}^{i\theta}}}{\cosh{(-\frac{\pi i}{2}+\rho\text{e}^{i\theta})}}i\rho\text{e}^{i\theta}\text{d}\theta
              \end{equation*}
              当 $\rho\to0$ 时,且 $\cosh{z}=\cos{iz}$,可以化简得到
              \begin{equation*}
                \oint_{C_2} \frac{\text{e}^z}{\cosh{z}}\text{d}z = 
                \int_{0}^{2\pi}\frac{\text{e}^{-\frac{\pi i}{2}}}{\cos{(\frac{\pi}{2}+i\rho\text{e}^{i\theta})}}i\rho\text{e}^{i\theta}\text{d}\theta = 
                \int_{0}^{2\pi}\frac{\text{e}^{-\frac{\pi i}{2}}}{-i\rho\text{e}^{i\theta}}i\rho\text{e}^{i\theta}\text{d}\theta =
                2\pi i
              \end{equation*}
              综上,最终得到
              \begin{equation*}
                \oint_{\left|z\right|=2} \frac{\text{e}^z}{\cosh{z}}\text{d}z =
                \oint_{C_1} \frac{\text{e}^z}{\cosh{z}}\text{d}z + 
                \oint_{C_2} \frac{\text{e}^z}{\cosh{z}}\text{d}z = 
                4\pi i.
              \end{equation*}
        \item 奇点为原点,在围道内,但不可以直接使用Cauchy积分公式,应根据Cauchy定理,将原积分围道转化为围绕原点的围道再求,
              在原点附近选取一半径为 $\rho$ 的圆为围道,
              不妨取 $z=\rho\text{e}^{i\theta}$,此时 $\text{d}z=i\rho\text{e}^{i\theta}\text{d}\theta$,则有
              \begin{equation*}
                \oint_{\left|z\right|=2} \frac{\sin{z}}{z^2}\text{d}z = 
                \int_{0}^{2\pi}\frac{\sin{(\rho\text{e}^{i\theta})}}{\rho^2\text{e}^{2i\theta}}i\rho\text{e}^{i\theta}\text{d}\theta = 
                \int_{0}^{2\pi}\frac{\sin{(\rho\text{e}^{i\theta})}}{\rho\text{e}^{i\theta}}i\text{d}\theta
              \end{equation*}
              当 $\rho\to0$ 时,可以化简得到
              \begin{equation*} 
                \oint_{\left|z\right|=2} \frac{\sin{z}}{z^2}\text{d}z = 
                \int_{0}^{2\pi}\frac{\rho\text{e}^{i\theta}}{\rho\text{e}^{i\theta}}i\text{d}\theta = 
                2\pi i.
              \end{equation*}
        \item 奇点为原点,在围道内,但不可以直接使用Cauchy积分公式,应根据Cauchy定理,将原积分围道转化为围绕原点的围道再求,
              在原点附近选取一半径为 $\rho$ 的圆为围道,
              不妨取 $z=\rho\text{e}^{i\theta}$,此时 $\text{d}z=i\rho\text{e}^{i\theta}\text{d}\theta$,则有
              \begin{equation*}
                \oint_{\left|z\right|=2} \frac{\left|z\right|\text{e}^z}{z^2}\text{d}z =
                \int_{0}^{2\pi}\frac{2\text{e}^{\rho\text{e}^{i\theta}}}{\rho^2\text{e}^{2i\theta}}i\rho\text{e}^{i\theta}\text{d}\theta =
                \int_{0}^{2\pi}\frac{2\text{e}^{\rho\text{e}^{i\theta}}}{\rho\text{e}^{i\theta}}i\text{d}\theta
              \end{equation*}
              当 $\rho\to0$ 时,可以化简得到
              \begin{equation*}
                \oint_{\left|z\right|=2} \frac{\left|z\right|\text{e}^z}{z^2}\text{d}z =
                \int_{0}^{2\pi} 2i\text{d}\theta = 
                4\pi i.
              \end{equation*}
        \item 由解析函数高阶导数公式 $\displaystyle f^{(n)}(z)=\frac{n!}{2\pi i}
              \oint_C\frac{f(\zeta)}{(\zeta - z)^{(n+1)}}\text{d}\zeta$,
              \begin{equation*}
                \oint_{\left|z\right|=2} \frac{\sin{z}}{z^4}\text{d}z = 
                \dfrac{2\pi i}{3!}\frac{\text{d}^3}{\text{d}z^3}(\sin{z})|_{z=0} =
                -\frac{\pi i}{3}.
              \end{equation*}
        \item 对被积函数分析,奇点为原点,对原式子进行拆分,得到
            \begin{equation*}
                f(z)=\frac{1}{z^2(z^2+16)}=\frac{1}{z^2}-\frac{15}{z^2+16}
            \end{equation*}
              显然拆分后后面分式无奇点,积分结果为0,前面分式积分结果也为0,故原积分结果为0.
    \end{enumerate}
\end{solution}
\begin{note}
    \begin{itemize}
        \item (4)需要注意,也可以用留数定理做,但不可以使用Cauchy积分公式。
        \item (5)也可以用解析函数高阶导数公式做,$\displaystyle 2\pi i\frac{\text{d}}{\text{d}z}(\sin{z})|_{z=0} = 2\pi i$。
        \item (6)也可以用解析函数高阶导数公式做,$\displaystyle 2\pi i\frac{\text{d}}{\text{d}z}(2\text{e}^z)|_{z=0} = 4\pi i$。
        \item \textcolor{red}{疑问:(7)如果按照缩小围道方法做,似乎无法得到正确答案?}
        \item (8)也可以用解析函数高阶导数公式做,$\displaystyle 2\pi i\frac{\text{d}}{\text{d}z}(\frac{1}{z^2+16})|_{z=0} = 0$。
    \end{itemize}
\end{note}

\newpage
\section{第四章习题}

\begin{problem}
    判断下列级数的收敛性与绝对收敛性:
    \begin{enumerate}[(1)]
        \item $\displaystyle\sum_{n=2}^\infty\frac{i^n}{\ln{n}}$;
        \item $\displaystyle\sum_{n=1}^\infty\frac{i^n}{n}$.
    \end{enumerate}
\end{problem}
\begin{solution}
    \begin{enumerate}[(1)]
        \item 对原级数进行拆分,
              \begin{equation*}
                \sum_{n=2}^\infty\frac{i^n}{\ln{n}} = 
                \sum_{k=1}^\infty\frac{(-1)^k}{\ln{2k}} + i\sum_{k=1}^\infty\frac{(-1)^k}{\ln{2k+1}}
              \end{equation*}
              由Leibnitz判别法可知,拆分后的两个交错级数都收敛,故原级数收敛,
              现判断是否绝对收敛:
              \begin{equation*}
                \left|\sum_{n=2}^\infty\frac{i^n}{\ln{n}}\right| = \sum_{n=2}^\infty\frac{1}{\ln{n}}
                > \sum_{n=2}^\infty\frac{1}{n}
              \end{equation*}
              调和级数发散,故 $\displaystyle\sum_{n=2}^\infty\frac{i^n}{\ln{n}}$ 收敛但不绝对收敛。
        \item 同(1)对原级数进行拆分
              \begin{equation*}
                \sum_{n=1}^\infty\frac{i^n}{n} = 
                \sum_{k=1}^\infty\frac{(-1)^k}{2k} + i\sum_{k=0}^\infty\frac{(-1)^k}{2k+1}
              \end{equation*}
              由Leibnitz判别法可知,拆分后的两个交错级数都收敛,故原级数收敛,
              现判断是否绝对收敛:
              \begin{equation*}
                \left|\sum_{n=1}^\infty\frac{i^n}{n}\right| = \sum_{n=1}^\infty\frac{1}{n}
              \end{equation*}
              调和级数发散,故 $\displaystyle\sum_{n=1}^\infty\frac{i^n}{n}$ 收敛但不绝对收敛。
    \end{enumerate}
\end{solution}

\begin{problem}
    试确定下列级数的收敛区域:
    \begin{enumerate}
        \item $\displaystyle\sum_{n=1}^\infty z^{n!}$;
        \item $\displaystyle\sum_{n=1}^\infty (\frac{z}{1+z})^n$;
        \item $\displaystyle\sum_{n=1}^\infty (-)^n(z^2+2z+2)^n$;
        \item $\displaystyle\sum_{n=1}^\infty 2^n\sin{\frac{z}{3^n}}$.
    \end{enumerate}
\end{problem}
\begin{solution}
    \begin{enumerate}[(1)]
        \item 对幂级数分析,有
              \begin{equation*}
                c_n = \left\{
                \begin{aligned}
                    &1 \quad n=k!,k=1,2,3,\cdots\\
                    &0 \quad others\\
                \end{aligned}
                \right
                .
              \end{equation*}
              根据Cauchy-Hadamard公式,收敛半径为
              \begin{equation*}
                R = \frac{1}{\varlimsup\limits_{n\to\infty}\left|c_n\right|^{\frac{1}{n}}}
                  = 1
              \end{equation*}
              收敛区域为 $\left|z\right|<1$;
        \item 进行换元,$\displaystyle t = \frac{z}{1+z}$,这时$c_n=1$,
              由Cauchy-Hadamard公式,收敛半径为1,故 $\left|\dfrac{z}{1+z}\right| < 1$,
              解出收敛区域为 $\text{Re}z > -\dfrac{1}{z}$;
        \item 进行换元,$t = z^2 + 2z + 2$,这时$c_n=(-)^n$,
              由Cauchy—Hadamard公式,收敛半径为1,故收敛区域为$\left|z^2 + 2z + 2\right| < 1$,
        \item 当 $n\to\infty$ 时,$\dfrac{z}{3^n}\to 0$在全平面成立,故该级数在全平面收敛。
    \end{enumerate}
\end{solution}
\begin{note}
    (3)收敛区域的数值求解没解出来。
\end{note}

\begin{problem}
    试求下列幂级数的收敛半径:
    \begin{enumerate}[(1)]
        \item $\displaystyle\sum_{n=1}^\infty \frac{1}{n^n}z^n$;
        \item $\displaystyle\sum_{n=1}^\infty \frac{1}{2^n n^n}z^n$;
        \item $\displaystyle\sum_{n=1}^\infty \frac{n!}{n^n}z^n$;
        \item $\displaystyle\sum_{n=1}^\infty \frac{(-)^n}{2^{2n}(n!)^2}z^n$;
        \item $\displaystyle\sum_{n=1}^\infty n^{\ln{n}}z^n$;
        \item $\displaystyle\sum_{n=1}^\infty \frac{1}{2^{2n}}z^{2n}$;
        \item $\displaystyle\sum_{n=1}^\infty \frac{\ln{n^n}}{n!}z^n$;
        \item $\displaystyle\sum_{n=1}^\infty (1-\frac{1}{n})^n z^n$.
    \end{enumerate}
\end{problem}
\begin{solution}
    \begin{enumerate}[(1)]
        \item $c_n = \dfrac{1}{n^n}$,根据Cauchy-Hadamard公式,收敛半径为
            \begin{equation*}
            R = \frac{1}{\varlimsup\limits_{n\to\infty}\left|c_n\right|^{\frac{1}{n}}}
              = \varliminf\limits_{n\to\infty}\left|\frac{1}{c_n}\right|^{\frac{1}{n}}
              = \varliminf\limits_{n\to\infty}\left|n^n\right|^{\frac{1}{n}}
              = \lim_{n\to\infty} n
              = \infty;
            \end{equation*}
        \item $c_n = \dfrac{1}{2^nn^n}$,根据Cauchy-Hadamard公式,收敛半径为
            \begin{equation*}
            R = \frac{1}{\varlimsup\limits_{n\to\infty}\left|c_n\right|^{\frac{1}{n}}}
              = \varliminf\limits_{n\to\infty}\left|\frac{1}{c_n}\right|^{\frac{1}{n}}
              = \varliminf\limits_{n\to\infty}\left|2^nn^n\right|^{\frac{1}{n}}
              = \lim_{n\to\infty} 2n
              = \infty;
            \end{equation*}
        \item $c_n = \dfrac{n!}{n^n}$,根据d'Alembert公式,收敛半径为
            \begin{equation*}
            R = \lim_{n\to\infty}\left|\frac{c_n}{c_{n+1}}\right|
              = \lim_{n\to\infty}\left|\frac{n!n^n}{(n+1)!(n+1)^{n+1}}\right|
              = \lim_{n\to\infty}(1+\frac{1}{n})^n
              = e;
            \end{equation*}
        \item $c_n = \frac{(-)^n}{2^{2n}(n!)^2}$,根据d'Alembert公式,收敛半径为
            \begin{equation*}
            R = \lim_{n\to\infty}\left|\frac{c_n}{c_{n+1}}\right|
              = \lim_{n\to\infty}\left|-\frac{2^{2(n+1)}[(n+1)!]^2}{2^{2n}(n!)^2}\right|
              = \lim_{n\to\infty}4(n+1)^2
              = \infty;
            \end{equation*}
        \item $c_n = n^{\ln{n}}$,根据Cauchy-Hadamard公式,收敛半径为
            \begin{equation*}
            R = \frac{1}{\varlimsup\limits_{n\to\infty}\left|c_n\right|^{\frac{1}{n}}}
              = \frac{1}{\varlimsup\limits_{n\to\infty}\left|n^{\ln{n}}\right|^{\frac{1}{n}}}
              = \lim_{n\to\infty}n^{\frac{\ln{n}}{n}}
              = 1;
            \end{equation*}
        \item 换元 $t = z^2$,此时
            \begin{equation*}
            c_n = \left\{
            \begin{aligned}
                &0 \ \qquad n=2k+1,k\in\mathbb{N}\\
                &\frac{1}{2^{2n}} \,\quad n=2k,k\in\mathbb{N}\\
            \end{aligned}
            \right
            .
            \end{equation*}
              根据Cauchy-Hadamard公式,对于$t$收敛半径为
              \begin{equation*}
                R = \frac{1}{\varlimsup\limits_{n\to\infty}\left|c_n\right|^{\frac{1}{n}}}
                  = \frac{1}{\varlimsup\limits_{n\to\infty}\left|2^{-2}\right|}
                  = 4
              \end{equation*}
               故$z$的收敛半径为2; 
        \item $c_n = \dfrac{n\ln{n}}{n!}$,根据d'Alembert公式,收敛半径为
            \begin{equation*}
            R = \lim_{n\to\infty}\left|\frac{c_n}{c_{n+1}}\right|
              = \lim_{n\to\infty}\left|-\frac{n\ln{n}}{\ln{(n+1)}}\right|
              = \infty;
            \end{equation*}
        \item $c_n = (1-\dfrac{1}{n})^n$,根据Cauchy-Hadamard公式,收敛半径为
        \begin{equation*}
        R = \frac{1}{\varlimsup\limits_{n\to\infty}\left|c_n\right|^{\frac{1}{n}}}
          = \frac{1}{\varlimsup\limits_{n\to\infty}\left|(1-\frac{1}{n})^n\right|^{\frac{1}{n}}}
          = \lim_{t\to\infty}1-\frac{1}{n}
          = 1.
        \end{equation*}
    \end{enumerate}
\end{solution}

\newpage
\section{第五章习题}

\begin{problem}
    将下列函数在指定点展开为Taylor级数,并给出其收敛半径:
    \begin{enumerate}[(1)]
        \item $1-z^2$,在 $z=1$ 展开;
        \item $\sin{z}$,在 $z=n\pi$ 展开;
        \item $\displaystyle\frac{1}{1+z+z^2}$,在 $z=0$ 展开;
        \item $\displaystyle\frac{\sin{z}}{1-z}$,在 $z=0$ 展开;
        \item $\displaystyle\text{e}^{\frac{1}{1-z}}$,在 $z=0$ 展开(可只求前四项).
    \end{enumerate}
\end{problem}
\begin{solution}
    \begin{enumerate}[(1)]
        \item $1 - z^2 = (1+z)(1-z) = (z-1)[-(z-1)-2] = -(z-1)^2 -2(z-1)$,在全平面收敛。
        \item 不妨取 $t = z - n\pi$,有 $\sin{z} = \sin{(t + n\pi)}$,\\[12pt]
              已知 $\displaystyle\sin{t} = \sum_{n = 0}^{\infty}\frac{(-)^n}{(2n+1)!}t^{2n+1}$,
              故 $\displaystyle\sin{(t + n\pi)} = \sum_{k = 0}^{\infty}\frac{(-)^{n+k}}{(2k+1)!}t^{2k+1}$,\\[12pt]
              即展开结果为 $\displaystyle\sin{z} = \sum_{k = 0}^{\infty}\frac{(-)^{n+k}}{(2k+1)!}(z-n\pi)^{2k+1}$,在全平面收敛。
        \item 因式分解得$\displaystyle\dfrac{1}{1+z+z^2} = \dfrac{1}{(z-\text{e}^{\frac{2\pi}{3}i})(z-\text{e}^{-\frac{2\pi}{3}i})} =
              \dfrac{1}{\sqrt{3}i}(\frac{\text{e}^{\frac{2\pi}{3}i}}{1-\text{e}^{\frac{2\pi}{3}i}}z - \frac{\text{e}^{-\frac{2\pi}{3}i}}{1-\text{e}^{-\frac{2\pi}{3}i}}z)$,\\[12pt]
              即展开结果为 $\displaystyle\dfrac{1}{\sqrt{3}i}\sum_{n=0}^{\infty}[\text{e}^{\frac{2(n+1)\pi}{3}}-\text{e}^{-\frac{2(n+1)\pi}{3}}]z^n =
              \frac{2}{\sqrt{3}}\sum_{n=0}^{\infty}\sin{[\frac{2}{3}(n+1)\pi]}·z^n$,
              收敛半径为1。
        \item $\displaystyle\frac{\sin{z}}{1-z} = \sin{z}·\frac{1}{1-z} = \sum_{k=0}^{\infty}\frac{(-)^k}{(2k+1)!}z^{2k+1}·\sum_{l=0}^{\infty}z^l =
              \sum_{k=0}^{\infty}\sum_{l=0}^{\infty}\frac{(-)^k}{(2k+1)!}z^{2k+l+1}$,\\[12pt]
              即展开结果为 $\displaystyle\sum_{n=1}^{\infty}(\sum_{k=0}^{\frac{n-1}{2}}\frac{(-)^k}{(2k+1)!})z^n$,收敛半径为1(公共区域)。
        \item 根据Taylor级数的定义,分别求出函数$\displaystyle f(x)=\text{e}^{\frac{1}{1-z}}$在$z=0$处的各阶导数,
              $f(0)=e,\ f'(0)=e,\ f^{(2)}(0)=3e,\ f^{(3)}(0)=13e,\ f^{(4)}(0)=73e$,故Taylor展开为$e+ez+\frac{3e}{2}z^2+\frac{13e}{6}z^3+\frac{73e}{24}z^4+\cdots$,
              收敛半径为1(最近的奇点为1)。
    \end{enumerate}
\end{solution}
\begin{note}
    \begin{itemize}
        \item (3)$\displaystyle\sum_{n=0}^{\infty}\frac{\sin{\frac{2(n+1)\pi}{3}}}{\sin{\frac{2\pi}{3}}}z^n$.
        \item (5)$\displaystyle\sum_{n=0}^{\infty}\frac{1}{n!}\frac{\text{d}^{n}(z^{n-1}\text{e}^{z})}{\text{d}z^n}\bigg|_{z=1}z^n$.
    \end{itemize}
\end{note}

\begin{problem}
    将下列函数在指定点展开为Taylor级数,并给出其收敛半径:
    \begin{enumerate}[(1)]
        \item $\ln{z}$,在 $z=i$ 展开,规定 $0\leq\text{arg}\ z<2\pi$;
        \item $\ln{z}$,在 $z=i$ 展开,规定 $\ln{z}|_{z=i}=-\dfrac{3}{2}\pi i$;
        \item $\arctan{z}$ 的主值,在 $z=0$ 展开;
        \item $\ln{\dfrac{1+z}{1-z}}$,在 $z=\infty$ 展开,规定 $\displaystyle\ln{\frac{1+z}{1-z}}|_{z=\infty}=(2k+1)\pi i$.
    \end{enumerate}
\end{problem}
\begin{solution}
    
\end{solution}

\begin{problem}
    求下列无穷级数之和,注意给出相应的收敛区域:
    \begin{enumerate}[(1)]
        \item $\displaystyle\sum_{n=0}^{\infty}\frac{1}{2n+1}z^{2n+1}$;
        \item $\displaystyle\sum_{n=0}^{\infty}\frac{1}{(2n)!}z^{2n}$;
        \item $\displaystyle\sum_{n=0}^{\infty}\sum_{m=0}^{\infty}\frac{(n+m)!}{n!\ m!}(\frac{z}{2})^{n+m}$;
        \item $\displaystyle\sum_{n=0}^{\infty}\sum_{m=0}^{\infty}\sum_{p=0}^{\infty}\frac{(n+m+p)!}{n!\ m!\ p!}(\frac{z}{3})^{n+m+p}$.
    \end{enumerate}
\end{problem}
\begin{solution}
    
\end{solution}

\begin{problem}
    求下列函数的Laurent展开:
    \begin{enumerate}[(1)]
    \item $\displaystyle\frac{1}{z^2(z-1)}$,在 $z=1$ 附近展开;
    \item $\displaystyle\frac{1}{z^2(z-1)}$,展开区域为 $1<\left|z\right|<\infty$;
    \item $\displaystyle\frac{1}{z^2-3z+2}$,展开区域为 $1<\left|z\right|<2$;
    \item $\displaystyle\frac{1}{z^2-3z+2}$,展开区域为 $2<\left|z\right|<\infty$;
    \item $\displaystyle\frac{(z-1)(z-2)}{(z-3)(z-4)}$,展开区域为 $3<\left|z\right|<4$;
    \item $\displaystyle\frac{(z-1)(z-2)}{(z-3)(z-4)}$,展开区域为 $4<\left|z\right|<\infty$;
    \end{enumerate}
\end{problem}
\begin{solution}
    
\end{solution}

\begin{problem}
    判断下列函数孤立奇点的性质,如果是极点,确定其阶数:
    \begin{enumerate}[(1)]
        \item $\dfrac{1}{z^2+a^2}$, $a\neq 0$;
        \item $\dfrac{\cos{az}}{z^2}$;
        \item $\displaystyle\frac{\cos{az}-\cos{bz}}{z^2}$, $a^2\neq b^2$;
        \item $\displaystyle\frac{\sin{z}}{z^2}-\frac{1}{z}$;
        \item $\displaystyle\cos{\frac{1}{\sqrt{z}}}$;
        \item $\displaystyle\frac{\sqrt{z}}{\sin{\sqrt{z}}}$;
        \item $\displaystyle\frac{1}{(z-1)\ln{z}}$;
        \item $\displaystyle\int_{0}^{z}\frac{\sinh{\sqrt{\zeta}}}{\sqrt{\zeta}}\text{d}\zeta$.
    \end{enumerate}
\end{problem}
\begin{solution}
    
\end{solution}

\newpage
\section{第六章习题}

\begin{problem}
    求下列函数在指定点 $z_0$ 处的留数:
    \begin{enumerate}[(1)]
        \item $\displaystyle\frac{1}{z-1}\text{e}^{z^2}$,\ $z_0=1$;
        \item $\displaystyle(\frac{z}{1-\cos{z}})^2$,\ $z_0=0$;
        \item $\displaystyle\frac{\text{e}^{z}}{(z^2-1)^2}$,\ $z_0=1$.
    \end{enumerate}
\end{problem}
\begin{solution}
    
\end{solution}

\begin{problem}
    求下列函数在复平面 $\mathbb{C}$ 内每一个孤立奇点处的留数:
    \begin{enumerate}[(1)]
        \item $\dfrac{1}{z^3-z^5}$;
        \item $\dfrac{z}{1-\cos{z}}$;
        \item $\displaystyle\text{e}^{\frac{1}{2}(z-\frac{1}{z})}$;
        \item $\displaystyle\frac{1}{(z-1)\ln{z}}$.
    \end{enumerate}
\end{problem}
\begin{solution}
    
\end{solution}

\begin{problem}
    计算下列积分值:
    \begin{enumerate}[(1)]
        \item $\displaystyle\oint_{\left|z-1\right|=1}\frac{1}{1+z^4}\ \text{d}z$;
        \item $\displaystyle\oint_{\left|z-1\right|=1}\frac{1}{z^2-1}\sin{\frac{\pi z}{4}}\ \text{d}z$;
        \item $\displaystyle\oint_{\left|z\right|=n}\tan{\pi z}\ \text{d}z$,$n$为正整数;
        \item $\displaystyle\oint_{\left|z\right|=1}\frac{\text{e}^z}{z^3}\ \text{d}z$;
    \end{enumerate}
\end{problem}
\begin{solution}
    
\end{solution}

\begin{problem}
    计算下列积分:
    \begin{enumerate}[(1)]
        \item $\displaystyle\int_{0}^{2\pi}\cos^{2n}{\theta}\ \text{d}\theta$,$n$为正整数;
        \item $\displaystyle\int_{0}^{\pi}\frac{\text{d}\theta}{1+\sin^2{\theta}}$.
    \end{enumerate}
\end{problem}
\begin{solution}
    
\end{solution}

\begin{problem}
    计算下列积分:
    \begin{enumerate}[(1)]
        \item $\displaystyle\int_{-\infty}^{\infty}\frac{x^2}{1+x^4}\text{d}x$;
        \item $\displaystyle\int_{-\infty}^{\infty}\frac{\text{d}x}{(1+x^2)\cosh{\frac{\pi x}{2}}}$.
    \end{enumerate}
\end{problem}
\begin{solution}
    
\end{solution}

\begin{problem}
    计算下列积分:
    \begin{enumerate}[(1)]
        \item $\displaystyle\int_{0}^{\infty}\frac{\cos{x}}{1+x^4}\text{d}x$;
        \item $\displaystyle\int_{0}^{\infty}\frac{\cos{x}}{(1+x^2)^3}\text{d}x$;
        \item $\displaystyle\int_{-\infty}^{\infty}\frac{x\sin{x}}{x^2-2x+2}\text{d}z$.
    \end{enumerate}
\end{problem}
\begin{solution}
    
\end{solution}

\begin{problem}
    计算下列积分:
    \begin{enumerate}[(1)]
        \item v.p.$\displaystyle\int_{-\infty}^{\infty}\frac{\text{d}x}{x(x-1)(x-2)}$;
        \item $\displaystyle\int_{0}^{\infty}\frac{x-\sin{x}}{x^3(1+x^2)}\text{d}x$;
        \item $\displaystyle\int_{-\infty}^{\infty}\frac{\text{e}^{px}-\text{e}^{qx}}{1-\text{e}^x}\text{d}x$,\ $0<p<1,0<q<1$.
    \end{enumerate}
\end{problem}
\begin{solution}
    
\end{solution}

\begin{problem}
    计算下列积分:
    \begin{enumerate}[(1)]
        \item v.p.$\displaystyle\int_{0}^{\infty}\frac{x^{s-1}}{1-x}\text{d}x$,\ $0<s<1$;
        \item $\displaystyle\int_{0}^{\infty}\frac{x^{s}}{(1+x^2)^2}\text{d}x$,\ $-1<s<3$;
        \item $\displaystyle\int_{0}^{\infty}\frac{x^{\alpha-1}\ln{x}}{1+x}\text{d}x$,\ $0<\alpha<1$;
        \item $\displaystyle\int_{0}^{\infty}\frac{\ln{x}}{(x+a)(x+b)}\text{d}x$,\ $b>a>0$.
    \end{enumerate}
\end{problem}

\newpage
\section{第七章习题}

\begin{problem}
    将下列连乘积用 $\Gamma$ 函数表示出来:
    \begin{enumerate}[(1)]
        \item $(2n)!!$;
        \item $(2n-1)!!$.
    \end{enumerate}
\end{problem}
\begin{solution}
    
\end{solution}

\begin{problem}
    计算下列积分:
    \begin{enumerate}[(1)]
        \item $\displaystyle\int_{0}^{\infty}x^{-\alpha}\sin{x}\text{d}x$, $0<\alpha<2$;
        \item $\displaystyle\int_{0}^{\infty}x^{-\alpha}\cos{x}\text{d}x$, $0<\alpha<1$.
    \end{enumerate}
\end{problem}
\begin{solution}

\end{solution}

\begin{problem}
    计算积分:$\displaystyle\int_{-1}^{1}(1-x)^p(1+x)^q\text{d}x$, $\text{Re}p>-1,\ \text{Re}q>-1$.
\end{problem}
\begin{solution}
    
\end{solution}

\newpage
\section{第八章习题}

\begin{problem}
    求下列函数的Laplace换式:
    \begin{enumerate}[(1)]
        \item $t^n$,$n=0,1,2,\cdots$;
        \item $t^{\alpha}$,$\text{Re}\alpha>-1$;
        \item $\displaystyle\text{e}^{\lambda t}\sin{\omega t}$,$\lambda>0,\ \omega>0$;
        \item $\displaystyle\int_{t}^{\infty}\frac{\cos{\tau}}{\tau}\text{d}\tau$.
    \end{enumerate}
\end{problem}
\begin{solution}
    
\end{solution}

\begin{problem}
    求下列Laplace换式的原函数:
    \begin{enumerate}[(1)]
        \item $\dfrac{a^3}{p(p+a)^3}$;
        \item $\dfrac{p^2+\omega^2}{(p^2-\omega^2)^2}$,$\omega>0$;
        \item $\displaystyle\frac{\text{e}^{-p\tau}}{p^2}$,$\tau>0$.
    \end{enumerate}
\end{problem}
\begin{solution}
    
\end{solution}

\begin{problem}
    利用Laplace变换计算积分:
    $\displaystyle\int_{0}^{\infty}\frac{\text{e}^{-ax}-\text{e}^{-bx}}{x}\cos{cx}\text{d}x$,
    $a>0,\ b>0,\ c>0$.
\end{problem}
\begin{solution}
    
\end{solution}

\begin{problem}
    用普遍反演公式求Laplace换式的原函数:
    $\displaystyle\frac{\text{e}^{-p\tau}}{p^4+4\omega^4}$,
    $\tau>0,\ \omega>0$.
\end{problem}
\begin{solution}
    
\end{solution}

\newpage
\section{第九章习题}

\begin{problem}
   求方程 $w'' - z^2w = 0$ 在 $z = 0$ 领域内的两个幂级数解。 
\end{problem}
\begin{solution}
    
\end{solution}

\begin{problem}
    求方程 $z^2(1-z)w'' + z(1-3z)w' - (1+z)w = 0$ 在 $z = 0$ 领域内的两个幂级数解。
\end{problem}
\begin{solution}
    
\end{solution}

\newpage
\section{第十章习题}

\begin{problem}
    证明 $\delta$ 函数的下列性质:
    \begin{enumerate}[(1)]
        \item $\delta(x) = \delta(-x)$;
        \item $x\delta(x) = 0$;
        \item $g(x)\delta(x) = g(0)\delta(x)$;
        \item $x\delta'(x) = -\delta(x)$;
        \item $\delta(ax) = \dfrac{1}{a}\delta(x)$, $a > 0$;
        \item $g(x)\delta'(x) = g(0)\delta'(x) - g'(x)\delta(x)$;
        \item $\delta(x^2 - a^2) = \dfrac{1}{2a}[\delta(x-a)+\delta(x+a)]$, $a>0$.
    \end{enumerate}
\end{problem}
\begin{solution}
    
\end{solution}

\newpage
\section{第十一章习题}

\begin{problem}
    在弦的横振动问题中,若弦受到一与速度成正比(比例系数为 $-\alpha$)的阻尼,试导出弦的有阻尼振动方程。
    又若除了阻尼力之外,弦还受到与弦的位移成正比(比例系数为 $-k$)的回复力,则此时弦的振动满足的方程是什么?
\end{problem}
\begin{solution}
    
\end{solution}

\begin{problem}
    一长为 $l$、横截面积为 $S$ 的均匀弹性杆,已知一端($x=0$)固定,另一端($x=l$)在杆轴方向上受拉力 $F$ 的作用而达到平衡。
    在 $t=0$ 时,撤去外力 $F$。试列出杆的纵振动所满足的方程、边界条件和初始条件。
\end{problem}
\begin{solution}
    
\end{solution}

\begin{problem}
    一长为 $l$ 的金属细杆(可近似地看成是一维的),通有稳定电流 $I$。如果杆的两端($x=0$ 和 $x=l$)均按Newton冷却定律与外界交换热量。
    外界温度为 $u_0$,初始时杆的温度为 $u_0(1-\dfrac{2x}{l})^2$。试写出杆上温度场所满足的方程、边界条件和初始条件,设金属的电阻为 $R$。
\end{problem}
\begin{solution}
    
\end{solution}

\begin{problem}
    在铀块中,除了中子的扩散运动外,还存在中子的吸收和增值过程。设在单位时间内、单位体积中吸收和增值的中子数均正比于该时刻、该处的中子浓度 $u(\boldsymbol{r},t)$,
    因而净增中子数可表为 $\alpha u(\boldsymbol{r},t)$, $\alpha$ 为比例常数。试导出 $u(\boldsymbol{r},t)$ 所满足的偏微分方程。
\end{problem}
\begin{solution}
    
\end{solution}
\end{document}